\begin{center}
\begin{threeparttable}[H]
\caption{アンケート結果の合計}
\label{table:questionnaire:sum}
\small
\begin{tabular}{|c|c|r|r|}
\hline
項目 & 推薦方法 & 悪い\tnote{a} & 良い\tnote{b} \\ \hline
\multirow{4}{*}{知名度} & 来店者新規度 & 67 & 13 \\ \cline{2-4}
 & やみつき度 & 60 & 20 \\ \cline{2-4}
 & 万人受け度 & 35 & 45 \\ \cline{2-4}
 & 食べログ上の点数 & 49 & 31 \\ \hline
\multirow{4}{*}{料理のクオリティ} & 来店者新規度 & 11 & 69 \\ \cline{2-4}
 & やみつき度 & 11 & 69 \\ \cline{2-4}
 & 万人受け度 & 11 & 69 \\ \cline{2-4}
 & 食べログ上の点数 & 10 & 70 \\ \hline
\multirow{4}{*}{コストパフォーマンス} & 来店者新規度 & 24 & 56 \\ \cline{2-4}
 & やみつき度 & 24 & 56 \\ \cline{2-4}
 & 万人受け度 & 9 & 71 \\ \cline{2-4}
 & 食べログ上の点数 & 31 & 49 \\ \hline
\multirow{4}{*}{行きたいと思うか} & 来店者新規度 & 24 & 56 \\ \cline{2-4}
 & やみつき度 & 24 & 56 \\ \cline{2-4}
 & 万人受け度 & 20 & 60 \\ \cline{2-4}
 & 食べログ上の点数 & 28 & 52 \\ \hline
\end{tabular}
\begin{tablenotes}
\item[a] 低い/絶対に行きたくない,またはやや低い/あまり行きたくないと評価された数
\item[b] やや高い/少し行ってみたい,または高い/かなり行ってみたいと評価された数
\end{tablenotes}
\end{threeparttable}
\end{center}
