\chapter{指標の評価}


% \newcommand{\ctext}[1]{\textcircled{\scriptsize #1}}
%丸囲み数字

\label{chap:experiment}

\section{目的}
%とりあえずお行儀は度外視で,内容だけ入れていきます.
提案する指標が,既存のサービスの推薦と比べて,より穴場な店を推薦することができることを示すために評価を行った.

\section{使用する来店履歴データ}
高松市内の飲食店に対して,実際に指標値を計算し指標が穴場具合を表していることを評価するため,高松市における2020年7月28日から2020年12月24日までに呟かれたSwarmのチェックインに連携したツイートをTwitter社の提供するSearch APIを用いて収集した.集めたツイートのうち,飲食店へのチェックインであるものを集めて得られた来店履歴を用いて高松市内の飲食店に対して指標値を計算した.

\section{評価方法}
提案する各指標値が高い飲食店上位10軒,および既存の飲食店推薦サービスである「食べログ」のランキング機能における2021年1月19日時点の高松市のランキング上位10軒からなる飲食店群に対して,食べログの評価機能とアンケートを用いて,飲食店の良質さと知名度を調査した.
ただし,高級店については,日常的に通うとは考えにくく,提案する指標では適切に評価できないことが明らかであるため,食べログ上での予算が一食5000円を超えるような飲食店はランキングから除外した.
	\subsection{食べログの評価機能を用いた飲食店の良質さと知名度の調査}\label{exp:scrutiny}

		上述の飲食店群に含まれる飲食店に対して食べログ上の点数と口コミの数とブックマークの数を調べ,表にまとめた.

	\subsection{アンケートによる良質さと知名度の調査}\label{exp:questionnaire}

		本実験では,飲食店の良質さが料理のクオリティ,およびコストパフォーマンスの二つの要素から決定されると仮定する.\par
		高松市在住の男子高専生8人に対して,以下の通りにアンケートを行った.ただし,アンケート対象者を無作為に2グループに分け,それぞれのグループに対して,ランダムな順で質問した.\\
		今まで行ったことがない飲食店に積極的に行ってみたいと思うかどうか尋ねた.\\
		穴場に注目した飲食店推薦サービスがあったら利用したいと思うかどうか尋ねた.\\
		外食の頻度について
		\begin{enumerate}
			\item 年に数回程度
			\item 月に数回程度
			\item 週に数回程度
			\item ほとんど毎日
		\end{enumerate}
		の4段階で尋ねた\\
		上述の飲食店群に含まれる飲食店について,\\
		この飲食店の高松市内での知名度について
		\begin{enumerate}
			\item 知らない,全く知られていない
			\item あまり知られていない
			\item そこそこ知られている
			\item 有名店である
		\end{enumerate}
		の4段階で尋ねた.\\
		料理の写真を提示し,料理のクオリティについて
		\begin{enumerate}
			\item 悪い
			\item やや悪い
			\item やや良い
			\item 良い
		\end{enumerate}
		の4段階で尋ねた.\\
		メニューの写真を提示し,コストパフォーマンスについて
		\begin{enumerate}
			\item 悪い
			\item やや悪い
			\item やや良い
			\item 良い
		\end{enumerate}
		の4段階で尋ねた.\\
		この飲食店に行きたいかどうかについて
		\begin{enumerate}
			\item 絶対に行きたくない
			\item あまり行きたくない
			\item 少し行ってみたい
			\item かなり行ってみたい
		\end{enumerate}
		の4段階で尋ねた.\\
\section{結果}
	\subsection{\ref{exp:scrutiny}の結果}
	% Please add the following required packages to your document preamble:
% \usepackage{multirow}
\begin{table}[H]
\centering
\caption{来店者新規度による推薦結果上位10軒}
\label{table:scrutiny:VN}
\small
\begin{tabular}{|c|r|r|r|r|r|r|}
\hline
\multirow{2}{*}{店名} & \multicolumn{3}{c|}{提案指標} & \multicolumn{3}{c|}{食べログ上の指標} \\ \cline{2-7}
 & \multicolumn{1}{c|}{来店者新規度} & \multicolumn{1}{c|}{やみつき度} & \multicolumn{1}{c|}{万人受け度} & \multicolumn{1}{c|}{点数} & \multicolumn{1}{c|}{口コミ数} & \multicolumn{1}{c|}{ブックマーク数} \\ \hline
炭火焼鳥 満天 & 0.143 & 3.513 & 1.000 & 3.07 & 4 & 79 \\ \hline
上海小籠包 & 0.200 & 2.342 & 1.000 & なし & なし & なし \\ \hline
つばめ家 & 0.200 & 2.342 & 1.000 & 3.30 & 11 & 427 \\ \hline
炭焼やき鳥 三吉 & 0.250 & 1.756 & 1.000 & 3.03 & 1 & 40 \\ \hline
武内食堂 鍛冶屋町店 & 0.250 & 1.756 & 1.000 & 3.08 & 4 & 94 \\ \hline
ヨコクラうどん & 0.250 & 3.212 & 0.875 & 3.62 & 54 & 1339 \\ \hline
ときわ食堂 & 0.250 & 1.756 & 1.000 & 3.00 & 4 & 38 \\ \hline
うどん市場 めんくい & 0.250 & 1.756 & 1.000 & 3.51 & 26 & 927 \\ \hline
Café Buono & 0.250 & 1.756 & 1.000 & 3.27 & 31 & 290 \\ \hline
Amazon & 0.250 & 4.792 & 0.833 & 3.15 & 6 & 96 \\ \hline
\end{tabular}
\end{table}

	% Please add the following required packages to your document preamble:
% \usepackage{multirow}
\begin{table}[H]
\centering
\caption{やみつき度による推薦結果上位10軒}
\label{table:scrutiny:addictivity}
\small
\begin{tabular}{|c|r|r|r|r|r|r|}
\hline
\multirow{2}{*}{店名} & \multicolumn{3}{c|}{提案指標} & \multicolumn{3}{c|}{食べログ上の指標} \\ \cline{2-7}
 & \multicolumn{1}{c|}{来店者新規度} & \multicolumn{1}{c|}{やみつき度} & \multicolumn{1}{c|}{万人受け度} & \multicolumn{1}{c|}{点数} & \multicolumn{1}{c|}{口コミ数} & \multicolumn{1}{c|}{ブックマーク数} \\ \hline
ホープ軒 福岡店 & 0.333 & 5.156 & 0.733 & 3.36 & 45 & 840 \\ \hline
手打十段 うどんバカ一代 & 0.890 & 4.887 & 2.684 & 3.79 & 1034 & 38204 \\ \hline
Amazon & 0.250 & 4.792 & 0.833 & 3.15 & 6 & 96 \\ \hline
OTTIMO イオンモール高松 & 0.560 & 3.538 & 2.880 & 3.04 & 3 & 24 \\ \hline
炭火焼鳥 満天 & 0.143 & 3.513 & 1.000 & 3.07 & 4 & 79 \\ \hline
ヨコクラうどん & 0.250 & 3.212 & 0.875 & 3.62 & 54 & 1339 \\ \hline
朔日 & 0.286 & 2.627 & 0.857 & 3.17 & 12 & 745 \\ \hline
手打ちうどん ますや & 0.375 & 2.451 & 0.750 & 3.67 & 44 & 2069 \\ \hline
上海小籠包 & 0.200 & 2.342 & 1.000 & なし & なし & なし \\ \hline
つばめ家 & 0.200 & 2.342 & 1.000 & 3.30 & 11 & 427 \\ \hline
\end{tabular}
\end{table}

	% Please add the following required packages to your document preamble:
% \usepackage{multirow}
\begin{table}[H]
\centering
\caption{万人受け度による推薦結果上位10軒}
\label{table:scrutiny:acceptability}
\small
\begin{tabular}{|c|r|r|r|r|r|r|}
\hline
\multirow{2}{*}{店名} & \multicolumn{3}{c|}{提案指標} & \multicolumn{3}{c|}{食べログ上の指標} \\ \cline{2-7}
 & \multicolumn{1}{c|}{来店者新規度} & \multicolumn{1}{c|}{やみつき度} & \multicolumn{1}{c|}{万人受け度} & \multicolumn{1}{c|}{点数} & \multicolumn{1}{c|}{口コミ数} & \multicolumn{1}{c|}{ブックマーク数} \\ \hline
めりけんや 高松駅前店 & 0.842 & 1.187 & 3.789 & 3.49 & 287 & 7275 \\ \hline
OTTIMO イオンモール高松 & 0.560 & 3.538 & 2.880 & 3.04 & 3 & 24 \\ \hline
手打十段 うどんバカ一代 & 0.890 & 4.886 & 2.684 & 3.79 & 1034 & 38204 \\ \hline
すき家 高松寿町店 & 0.400 & 1.283 & 2.400 & なし & なし & なし \\ \hline
裏きせき & 0.529 & 1.503 & 2.353 & 3.37 & 21 & 686 \\ \hline
海鮮食堂 じゃこや & 0.563 & 1.139 & 2.000 & 3.51 & 76 & 2741 \\ \hline
ダントツラーメン 岡山一番店 & 0.615 & 0.941 & 1.231 & 3.28 & 34 & 696 \\ \hline
しんみょう精肉店 鍛冶屋町店 & 0.400 & 0.970 & 1.200 & 3.04 & 3 & 243 \\ \hline
吉野家 高松瓦町店 & 0.400 & 0.970 & 1.200 & なし & なし & なし \\ \hline
麺屋 がんてつ & 0.667 & 1.791 & 1.143 & 3.38 & 52 & 703 \\ \hline
\end{tabular}
\end{table}

	% Please add the following required packages to your document preamble:
% \usepackage{multirow}
\begin{table}[H]
\centering
\caption{食べログ上の点数による推薦結果上位10軒}
\label{table:scrutiny:rank}
\small
\begin{tabular}{|c|S|S|S|S|S|S|}
\hline
\multirow{2}{*}{店名} & \multicolumn{3}{c|}{提案指標} & \multicolumn{3}{c|}{食べログ上の指標} \\ \cline{2-7}
 & \multicolumn{1}{c|}{来店者新規度} & \multicolumn{1}{c|}{やみつき度} & \multicolumn{1}{c|}{万人受け度} & \multicolumn{1}{c|}{点数} & \multicolumn{1}{c|}{口コミ数} & \multicolumn{1}{c|}{ブックマーク数} \\ \hline
瀬戸晴れ & 0.750 & 0.290 & 0.500 & 3.94 & 77 & 3685 \\ \hline
うどん 一福 & 0.882 & 0.140 & 0.235 & 3.91 & 493 & 26692 \\ \hline
手打うどん 麦蔵 & 1.000 & 0.000 & 0.333 & 3.89 & 200 & 16172 \\ \hline
手打ちうどん はりや & 1.000 & 0.000 & 0.143 & 3.87 & 178 & 8878 \\ \hline
手打ちうどん 大蔵 & 1.000 & 0.000 & 0.500 & 3.85 & 51 & 2540 \\ \hline
うどん本陣 山田家 讃岐本店 & 0.952 & 0.059 & 0.193 & 3.82 & 452 & 20482 \\ \hline
ふる里うどん & なし & なし & なし & 3.81 & 40 & 1551 \\ \hline
手打十段 うどんバカ一代 & 0.890 & 4.886 & 2.684 & 3.79 & 1034 & 38204 \\ \hline
さか枝 & 0.909 & 0.109 & 0.182 & 3.79 & 540 & 20382 \\ \hline
竹清 & 0.912 & 0.100 & 0.167 & 3.78 & 423 & 20455 \\ \hline
\end{tabular}
\end{table}

	\subsection{\ref{exp:questionnaire}の結果}
	% Please add the following required packages to your document preamble:
% \usepackage{multirow}
\begin{table}[H]
\centering
\caption{来店者新規度による推薦のアンケート結果抜粋}
\label{table:questionnaire:VN}
\small
\scalebox{0.7}[1.0]{
\begin{tabular}{|l|l|l|l|l|l|}
\hline
店名 & 項目 & 低い/絶対に行きたくない & やや低い/あまり行きたくない & やや高い/少し行ってみたい & 高い/かなり行ってみたい \\ \hline
\multirow{4}{*}{炭火焼鳥 満天} & 知名度 & 5 & 1 & 1 & 1 \\ \cline{2-6}
 & 料理のクオリティ & 0 & 1 & 3 & 4 \\ \cline{2-6}
 & コストパフォーマンス & 0 & 4 & 4 & 0 \\ \cline{2-6}
 & 行きたいと思うか & 1 & 1 & 3 & 3 \\ \hline
\multirow{4}{*}{上海小籠包} & 知名度 & 4 & 2 & 2 & 0 \\ \cline{2-6}
 & 料理のクオリティ & 0 & 0 & 3 & 5 \\ \cline{2-6}
 & コストパフォーマンス & 0 & 0 & 5 & 3 \\ \cline{2-6}
 & 行きたいと思うか & 0 & 0 & 4 & 4 \\ \hline
\multirow{4}{*}{つばめ家} & 知名度 & 6 & 0 & 2 & 0 \\ \cline{2-6}
 & 料理のクオリティ & 0 & 1 & 3 & 4 \\ \cline{2-6}
 & コストパフォーマンス & 1 & 3 & 3 & 1 \\ \cline{2-6}
 & 行きたいと思うか & 0 & 4 & 3 & 1 \\ \hline
\end{tabular}
}
\end{table}

	% Please add the following required packages to your document preamble:
% \usepackage{multirow}
\begin{table}[H]
\centering
\caption{やみつき度による推薦のアンケート結果}
\label{table:questionnaire:addictivity}
\small
\begin{tabular}{|c|c|r|r|r|r|}
\hline
店名 & 項目 & \multicolumn{1}{Wc{4em}|}{低い} & \multicolumn{1}{Wc{4em}|}{やや低い} & \multicolumn{1}{Wc{4em}|}{やや高い} & \multicolumn{1}{Wc{4em}|}{高い} \\ \hline
\multirow{4}{*}{\begin{tabular}[c]{@{}c@{}}ホープ軒 \\ 福岡店\end{tabular}} & 知名度 & 3 & 1 & 2 & 2 \\ \cline{2-6} 
 & 料理のクオリティ & 0 & 1 & 4 & 3 \\ \cline{2-6} 
 & コストパフォーマンス & 1 & 3 & 3 & 1 \\ \cline{2-6} 
 & 行きたいと思うか & 0 & 2 & 4 & 2 \\ \hline
\multirow{4}{*}{\begin{tabular}[c]{@{}c@{}}手打十段 \\ うどんバカ一代\end{tabular}} & 知名度 & 3 & 1 & 1 & 3 \\ \cline{2-6} 
 & 料理のクオリティ & 0 & 0 & 5 & 3 \\ \cline{2-6} 
 & コストパフォーマンス & 0 & 1 & 7 & 0 \\ \cline{2-6} 
 & 行きたいと思うか & 0 & 1 & 5 & 2 \\ \hline
\multirow{4}{*}{Amazon} & 知名度 & 7 & 1 & 0 & 0 \\ \cline{2-6} 
 & 料理のクオリティ & 2 & 2 & 4 & 0 \\ \cline{2-6} 
 & コストパフォーマンス & 2 & 1 & 4 & 1 \\ \cline{2-6} 
 & 行きたいと思うか & 2 & 6 & 0 & 0 \\ \hline
\multirow{4}{*}{\begin{tabular}[c]{@{}c@{}}OTTIMO \\ イオンモール高松\end{tabular}} & 知名度 & 4 & 3 & 1 & 0 \\ \cline{2-6} 
 & 料理のクオリティ & 0 & 0 & 6 & 2 \\ \cline{2-6} 
 & コストパフォーマンス & 0 & 3 & 5 & 0 \\ \cline{2-6} 
 & 行きたいと思うか & 0 & 2 & 5 & 1 \\ \hline
\multirow{4}{*}{\begin{tabular}[c]{@{}c@{}}炭火焼鳥 \\ 満天\end{tabular}} & 知名度 & 5 & 1 & 1 & 1 \\ \cline{2-6} 
 & 料理のクオリティ & 0 & 1 & 3 & 4 \\ \cline{2-6} 
 & コストパフォーマンス & 0 & 4 & 4 & 0 \\ \cline{2-6} 
 & 行きたいと思うか & 1 & 1 & 3 & 3 \\ \hline
\multirow{4}{*}{ヨコクラうどん} & 知名度 & 5 & 1 & 2 & 0 \\ \cline{2-6} 
 & 料理のクオリティ & 0 & 1 & 6 & 1 \\ \cline{2-6} 
 & コストパフォーマンス & 0 & 1 & 4 & 3 \\ \cline{2-6} 
 & 行きたいと思うか & 0 & 2 & 4 & 2 \\ \hline
\multirow{4}{*}{朔日} & 知名度 & 7 & 0 & 1 & 0 \\ \cline{2-6} 
 & 料理のクオリティ & 0 & 1 & 3 & 4 \\ \cline{2-6} 
 & コストパフォーマンス & 1 & 2 & 4 & 1 \\ \cline{2-6} 
 & 行きたいと思うか & 0 & 1 & 6 & 1 \\ \hline
\multirow{4}{*}{\begin{tabular}[c]{@{}c@{}}手打ちうどん \\ ますや\end{tabular}} & 知名度 & 4 & 2 & 1 & 1 \\ \cline{2-6} 
 & 料理のクオリティ & 0 & 2 & 3 & 3 \\ \cline{2-6} 
 & コストパフォーマンス & 0 & 1 & 5 & 2 \\ \cline{2-6} 
 & 行きたいと思うか & 0 & 2 & 4 & 2 \\ \hline
\multirow{4}{*}{上海小籠包} & 知名度 & 4 & 2 & 2 & 0 \\ \cline{2-6} 
 & 料理のクオリティ & 0 & 0 & 3 & 5 \\ \cline{2-6} 
 & コストパフォーマンス & 0 & 0 & 5 & 3 \\ \cline{2-6} 
 & 行きたいと思うか & 0 & 0 & 4 & 4 \\ \hline
\multirow{4}{*}{つばめ家} & 知名度 & 6 & 0 & 2 & 0 \\ \cline{2-6} 
 & 料理のクオリティ & 0 & 1 & 3 & 4 \\ \cline{2-6} 
 & コストパフォーマンス & 1 & 3 & 3 & 1 \\ \cline{2-6} 
 & 行きたいと思うか & 0 & 4 & 3 & 1 \\ \hline
\end{tabular}
\end{table}

	% Please add the following required packages to your document preamble:
% \usepackage{multirow}
\begin{table}[H]
\centering
\caption{万人受け度による推薦のアンケート結果抜粋}
\label{table:questionnaire:acceptability}
\small
\scalebox{0.7}[1.0]{
\begin{tabular}{|l|l|l|l|l|l|}
\hline
店名 & 項目 & 低い/絶対に行きたくない & やや低い/あまり行きたくない & やや高い/少し行ってみたい & 高い/かなり行ってみたい \\ \hline
\multirow{4}{*}{めりけんや 高松駅前店} & 知名度 & 3 & 1 & 3 & 1 \\ \cline{2-6}
 & 料理のクオリティ & 0 & 3 & 4 & 1 \\ \cline{2-6}
 & コストパフォーマンス & 0 & 1 & 4 & 3 \\ \cline{2-6}
 & 行きたいと思うか & 0 & 3 & 4 & 1 \\ \hline
\multirow{4}{*}{OTTIMO イオンモール高松} & 知名度 & 4 & 3 & 1 & 0 \\ \cline{2-6}
 & 料理のクオリティ & 0 & 0 & 6 & 2 \\ \cline{2-6}
 & コストパフォーマンス & 0 & 3 & 5 & 0 \\ \cline{2-6}
 & 行きたいと思うか & 0 & 2 & 5 & 1 \\ \hline
\multirow{4}{*}{手打十段 うどんバカ一代} & 知名度 & 3 & 1 & 1 & 3 \\ \cline{2-6}
 & 料理のクオリティ & 0 & 0 & 5 & 3 \\ \cline{2-6}
 & コストパフォーマンス & 0 & 1 & 7 & 0 \\ \cline{2-6}
 & 行きたいと思うか & 0 & 1 & 5 & 2 \\ \hline
\end{tabular}
}
\end{table}

	% Please add the following required packages to your document preamble:
% \usepackage{multirow}
\begin{table}[H]
\centering
\caption{食べログ上の点数による推薦のアンケート結果}
\label{table:questionnaire:rank}
\small
\begin{tabular}{|c|c|r|r|r|r|}
\hline
店名 & 項目 & \multicolumn{1}{Wc{4em}|}{低い} & \multicolumn{1}{Wc{4em}|}{やや低い} & \multicolumn{1}{Wc{4em}|}{やや高い} & \multicolumn{1}{Wc{4em}|}{高い} \\ \hline
\multirow{4}{*}{おうどん 瀬戸晴れ} & 知名度 & 5 & 1 & 1 & 1 \\ \cline{2-6}
 & クオリティ & 0 & 0 & 2 & 6 \\ \cline{2-6}
 & コストパフォーマンス & 1 & 3 & 4 & 0 \\ \cline{2-6}
 & 行きたいと思うか & 0 & 2 & 3 & 3 \\ \hline
\multirow{4}{*}{うどん 一福} & 知名度 & 3 & 1 & 3 & 1 \\ \cline{2-6}
 & 料理のクオリティ & 0 & 1 & 5 & 2 \\ \cline{2-6}
 & コストパフォーマンス & 0 & 1 & 2 & 5 \\ \cline{2-6}
 & 行きたいと思うか & 0 & 3 & 4 & 1 \\ \hline
\multirow{4}{*}{手打うどん 麦蔵} & 知名度 & 5 & 1 & 2 & 0 \\ \cline{2-6}
 & 料理のクオリティ & 0 & 3 & 4 & 1 \\ \cline{2-6}
 & コストパフォーマンス & 2 & 5 & 1 & 0 \\ \cline{2-6}
 & 行きたいと思うか & 1 & 5 & 2 & 0 \\ \hline
\multirow{4}{*}{手打うどん はりや} & 知名度 & 4 & 1 & 3 & 0 \\ \cline{2-6}
 & 料理のクオリティ & 0 & 0 & 2 & 6 \\ \cline{2-6}
 & コストパフォーマンス & 3 & 0 & 4 & 1 \\ \cline{2-6}
 & 行きたいと思うか & 1 & 0 & 3 & 4 \\ \hline
\multirow{4}{*}{手打ちうどん大蔵} & 知名度 & 5 & 1 & 1 & 1 \\ \cline{2-6}
 & 料理のクオリティ & 0 & 0 & 1 & 7 \\ \cline{2-6}
 & コストパフォーマンス & 1 & 4 & 3 & 0 \\ \cline{2-6}
 & 行きたいと思うか & 1 & 2 & 4 & 1 \\ \hline
\multirow{4}{*}{うどん本陣 山田家} & 知名度 & 3 & 0 & 1 & 4 \\ \cline{2-6}
 & 料理のクオリティ & 1 & 0 & 0 & 7 \\ \cline{2-6}
 & コストパフォーマンス & 2 & 2 & 3 & 1 \\ \cline{2-6}
 & 行きたいと思うか & 0 & 3 & 3 & 2 \\ \hline
\multirow{4}{*}{ふる里うどん} & 知名度 & 5 & 1 & 1 & 1 \\ \cline{2-6}
 & 料理のクオリティ & 0 & 0 & 8 & 0 \\ \cline{2-6}
 & コストパフォーマンス & 0 & 3 & 4 & 1 \\ \cline{2-6}
 & 行きたいと思うか & 0 & 3 & 4 & 1 \\ \hline
\multirow{4}{*}{手打十段 うどんバカ一代} & 知名度 & 3 & 1 & 1 & 3 \\ \cline{2-6}
 & 料理のクオリティ & 0 & 0 & 5 & 3 \\ \cline{2-6}
 & コストパフォーマンス & 0 & 1 & 7 & 0 \\ \cline{2-6}
 & 行きたいと思うか & 0 & 1 & 5 & 2 \\ \hline
\multirow{4}{*}{さか枝うどん} & 知名度 & 2 & 0 & 2 & 4 \\ \cline{2-6}
 & 料理のクオリティ & 0 & 3 & 1 & 4 \\ \cline{2-6}
 & コストパフォーマンス & 0 & 2 & 1 & 5 \\ \cline{2-6}
 & 行きたいと思うか & 0 & 3 & 4 & 1 \\ \hline
\multirow{4}{*}{竹清 本店} & 知名度 & 4 & 3 & 0 & 1 \\ \cline{2-6}
 & 料理のクオリティ & 0 & 2 & 5 & 1 \\ \cline{2-6}
 & コストパフォーマンス & 0 & 1 & 2 & 5 \\ \cline{2-6}
 & 行きたいと思うか & 1 & 2 & 3 & 2 \\ \hline
\end{tabular}
\end{table}

