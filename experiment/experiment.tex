\chapter{指標の評価実験}

\newcommand{\ctext}[1]{\textcircled{\scriptsize #1}}
%丸囲み数字

\label{chap:experiment}

\section{目的}
%とりあえずお行儀は度外視で,内容だけ入れていきます.
		\begin{enumerate}
			\item 穴場店の推薦はユーザに必要とされている
			\item 提案する指標が,既存のサービスや,人気に基づいた単純な推薦と比べて,より穴場な店を推薦することができる
		\end{enumerate}
以上の二つを示すために評価実験を行った.
\section{実験方法}

次のように飲食店群\ctext{1},\ctext{2}を設定する.\\
	\ctext{1}:提案する指標値が高い飲食店上位10軒,および既存の飲食店推薦サービスである「食べログ」のランキング機能における2021年1月19日時点の高松市のランキング上位10軒の飲食店\\
	\ctext{2}:\ctext{1}に加えて,使用したデータにおいてべ来店者数が多い飲食店上位10軒\par
	この二つの飲食店群の要素に対して,食べログの評価機能とアンケートを用いて,飲食店の良質さと知名度を実施した.ただし,高級店については,日常的に通うとは考えにくく,提案する指標では適切に評価できないことが明らかであるため,食べログ上での予算が一食5000円を超えるような飲食店は除外した.
	\subsection{食べログの評価機能を用いた飲食店の良質さと知名度の調査}\label{exp:scrutiny}

		上述の飲食店群\ctext{2}の食べログ上の点数と口コミの数とブックマークの数を調べ,表にまとめた.

	\subsection{アンケートによる良質さと知名度の調査}\label{exp:questionnaire}

		本実験では,飲食店の良質さが料理のクオリティ,およびコストパフォーマンスの二つの要素から決定されると仮定する.\par
		% 緒言で,良質さについて,この二つで決まると断定していいような定義をする.
		高松市在住の男子高専生7人に対してアンケートを行った.\\
		今まで行ったことがない飲食店に積極的に行ってみたいと思うかどうか尋ねた.\\
		穴場に注目した飲食店推薦サービスがあったら利用したいと思うかどうか尋ねた.\\
		外食の頻度について
		\begin{enumerate}
			\item 年に数回程度
			\item 月に数回程度
			\item 週に数回程度
			\item ほとんど毎日
		\end{enumerate}
		の4段階で尋ねた\\
		上述の飲食店群\ctext{1}の各要素について,\\
		この飲食店が高松市民にとって知名度が高いと思うかについて
		\begin{enumerate}
			\item 知らない,全く知られていない
			\item あまり知られていない
			\item そこそこ知られている
			\item 有名店である
		\end{enumerate}
		の4段階で尋ねた.\\
		料理の写真を提示し,料理のクオリティについて
		\begin{enumerate}
			\item 悪い
			\item やや悪い
			\item やや良い
			\item 良い
		\end{enumerate}
		の4段階で尋ねた.\\
		メニューの写真を提示し,コストパフォーマンスについて
		\begin{enumerate}
			\item 悪い
			\item やや悪い
			\item やや良い
			\item 良い
		\end{enumerate}
		の4段階で尋ねた.\\
		この飲食店に行きたいかどうかについて
		\begin{enumerate}
			\item 絶対に行きたくない
			\item あまり行きたくない
			\item 少し行ってみたい
			\item かなり行ってみたい
		\end{enumerate}
		の4段階で尋ねた.\\
\section{実験結果}
	\subsection{\ref{exp:scrutiny}の結果}
	\subsection{\ref{exp:questionnaire}の結果}
\section{結果の考察}
