\chapter{緒言}
\label{chap:intro}
身近な飲食店推薦サービスである食べログ\footnote{https://tabelog.com}やRetty\footnote{https://retty.me}では,有志のユーザによる評価をもとにした,集合知による推薦が行われている.
しかし,集合知による推薦では,知名度が低い飲食店は,サービスや料理の質に関わらず,高い点数や評価を得られない.
特に食べログでは,評価が集まらないと点数が上がらないことが明言されている\cite{score}.
すなわち,新規オープンした飲食店や熱心な集客活動を行っていない飲食店は,その質によらず,点数や評価が低くなりやすい.
したがって,知名度が高い店は推薦されやすく,多数の評価を得られるのに対し,知名度が低い店は推薦されにくく,なかなか評価を得ることができない.
これにより,推薦される飲食店が固定され,知名度の差が開き続けていく問題や,知名度が低い店を発見しにくいという問題が生じる.
この問題を解消するために,知名度が低く,かつ良質な飲食店への推薦が必要であると考える.本研究において「知名度が低く,かつ良質な飲食店」を穴場飲食店と呼ぶこととする.

本研究では,飲食店が穴場である可能性を示す指標を考案することで,穴場飲食店への推薦を実現する.
飲食店が穴場である可能性の計算には来店履歴データを用いる.新規来店者が少ない飲食店,ごく少数の熱心なリピーターがいるがのべ来店者数が少ない飲食店,および複数のユーザが何度も来店しているがのべ来店者数は少ない店は穴場である可能性が高い.
%「穴場飲食店には少数のユーザが頻繁に通い,その他のユーザはその存在に気づいていない」という仮定のもと,ある飲食店に誰が何度訪れたかを集めた来店履歴データを利用した指標を考案する.
%さらに,近年では,スマートフォンの普及に伴い,多くの人々がSNS上で手軽に様々な情報を共有するようになった.
%その中でも,位置情報を共有するSNSであるSwarmは,スマートフォンのGPS機能を使って,自分が現在どこにいるかを「チェックイン」という形で記録,共有できる.また,Twitterと連携し,チェックイン情報をツイートすることで,ライフログを公開データとして記録することができる.
%本研究では,Twitter上のSwarmのチェックインに連携したツイートをもとに,来店履歴データを作成し,高松市内の飲食店に対して指標値を実際に計算する.
提案する指標をもとに推薦された飲食店と既存サービスにより推薦された飲食店が穴場かどうかを,既存サービス上の指標やアンケートによる知名度と良質さの調査によって比較し,指標が穴場を推薦できているかどうか評価する.
