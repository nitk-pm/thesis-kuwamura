\chapter{緒言}
\label{chap:intro}
身近な飲食店推薦サービスである食べログ[N]やRetty[N]では,有志のユーザによるレビューをもとに飲食店の良し悪しを評価する,集合知による推薦が行われている.
しかし,この方式では,飲食店の質に関わらず,知名度が低い場合は高い点数を得られない.
特に食べログでは,評価の数が少ない場合,点数が低くなることが明言されている[N].
新規オープンした飲食店や熱心な集客活動を行っていない飲食店は,その質によらず,知名度や評価数の点で不利である.
したがって,知名度が高い店は推薦されやすく,多数の評価を得られるのに対し,知名度が低い店は推薦されにくく,なかなか評価を得ることができないという問題が生じる
.この問題を解消するため,良質だが知名度が低い穴場飲食店への推薦が求められている.

飲食店が穴場であるという根拠になるのが,少数のリピーターによる頻繁な来店である.
本研究では,穴場飲食店とは「少数のユーザが頻繁に通い,その他のユーザはその存在に気づいていない飲食店」であるという仮定のもと,来店履歴データを利用した指標を考案する.
また,本研究では,リクルートライフスタイルによる「飲食店リピート実態&リピート要因調査」[N]をもとに,本研究における「飲食店の良質さ」を定め,指標の評価に用いる.

さらに,近年では,スマートフォンの普及に伴い,多くの人々がSNS上で手軽に様々な情報を共有するようになった.
その中でも,位置情報を共有するSNSであるSwarm[N]は,スマートフォンのGPS機能を使って,自分が現在どこにいるかを「チェックイン」という形で記録,共有できる.
また,Twitter[N]と連携し,チェックイン情報をツイートすることで,ライフログを公開データとして記録することができる.
本研究では,Twitter上のSwarmのチェックインに連携したツイートをもとに来店履歴データを作成し,高松市内の飲食店に対して指標値を実際に計算する.
得られた指標値をもとに推薦された飲食店と既存サービスにより推薦された飲食店に対して,良質さと知名度をアンケートで比較し,指標が穴場を推薦できているかの評価を行う.
