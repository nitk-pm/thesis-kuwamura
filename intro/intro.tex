\chapter{緒言}
\label{chap:intro}
身近な飲食店推薦サービスである食べログやRettyでは,有志のユーザによる評価をもとにした,集合知による推薦が行われている.
しかし,集合知による推薦では,知名度が低い飲食店は,サービスや料理の質に関わらず,高い点数や評価得られない.
特に食べログでは,評価が集まらないと点数が上がらないことが明言されている\cite{score}.
すなわち,新規オープンした飲食店や熱心な集客活動を行っていない飲食店は,その質によらず,点数や評価が低くなりやすい.
したがって,知名度が高い店は推薦されやすく,多数の評価を得られるのに対し,知名度が低い店は推薦されにくく,なかなか評価を得ることができない.これにより,推薦される飲食店が固定され,知名度の差が開き続けていく問題や,知名度が低い店を発見しにくいという問題が生じる.
この問題を解消するために,知名度が低く,かつ良質な飲食店への推薦が必要であると考える.本研究において「知名度が低く,かつ良質な飲食店」を穴場飲食店と呼ぶこととする.

本研究では,飲食店がどれぐらい穴場かを示す指標を考案することで,穴場飲食店への推薦を実現する.
飲食店が穴場であるという根拠には,少数のリピーターによる頻繁な来店を用いる.
「穴場飲食店には少数のユーザが頻繁に通い,その他のユーザはその存在に気づいていない」という仮定のもと,飲食店への来店履歴を利用した指標を考案する.
%さらに,近年では,スマートフォンの普及に伴い,多くの人々がSNS上で手軽に様々な情報を共有するようになった.
%その中でも,位置情報を共有するSNSであるSwarmは,スマートフォンのGPS機能を使って,自分が現在どこにいるかを「チェックイン」という形で記録,共有できる.また,Twitterと連携し,チェックイン情報をツイートすることで,ライフログを公開データとして記録することができる.
%本研究では,Twitter上のSwarmのチェックインに連携したツイートをもとに,来店履歴データを作成し,高松市内の飲食店に対して指標値を実際に計算する.

提案する指標をもとに推薦された飲食店と既存サービスにより推薦された飲食店に対して,穴場具合をアンケートで比較し,指標が穴場を推薦できているかの評価を行う.
