\chapter{関連研究}

% \section{穴場への推薦に関する関連研究}
穴場への推薦に関連する研究として,写真共有サイトを用いた穴場撮影スポットの抽出を扱った研究がある.\par
西脇ら\cite{picture}は,撮影スポットの穴場具合を示す指標である,穴場スポット度を提案している.穴場スポット度は写真共有サイト内における,ある撮影スポットで撮られた写真のお気に入り数$F$とそのスポットから撮影した人の数$|C|$を用いて,$\frac{\log_{10}F}{|C|}$と表される.西脇らはこの指標を飲食店への推薦に応用可能であるとしているが,実際に応用した例はない.本研究では穴場スポット度に代わる,穴場飲食店推薦のための三つの指標を提案する.\par
また,穴場への推薦に関連した研究として,セレンディピティの高い推薦の研究が挙げられる.\par
食べログのような,多くの評価を必要とする推薦は良質な飲食店を推薦する可能性が高い.しかし,McNeeら\cite{McNee}はこのような精度のみを重視した推薦がユーザにとって有用とは限らないということを指摘している.
本研究が実現を目指す穴場飲食店への推薦のように,ユーザに新たな発見を促すような推薦は,セレンディピティが高いといわれる.推薦におけるセレンディピティとは,「思いがけないものを推薦すること」であり\cite{McNee},セレンディピティが高い推薦によって,他の方法では発見できない意外なものを推薦できる\cite{Parameswaran}.\par
加藤ら\cite{onomatopoeia}は,飲食店の口コミに含まれるオノマトペを用いてセレンディピティのある推薦方法を提案している.食感や見た目などの要素を表すオノマトペから飲食店の類似度を計算し,ユーザに対して重視する要素を選ばせて推薦を行う.しかし,この方法では,口コミにオノマトペを含まない飲食店を推薦できない.知名度が低い飲食店は口コミの数が少ないと予想され,そういった飲食店を推薦するためには新たな手法が必要となる.本研究では口コミが少ない飲食店を推薦可能な指標を提案する.
% \section{ユーザの行動履歴の利用に関連する研究}
% ユーザの行動履歴をもとに,飲食店が穴場である可能性を調べた例はない.
