\chapter{関連研究}

% \section{穴場への推薦に関する関連研究}
食べログのような,多くの評価を必要とする推薦は良質な飲食店を推薦する可能性が高い.しかし,このような精度のみを重視した推薦がユーザにとって有用とは限らないということが指摘している\cite{McNee}.
本研究が実現を目指す穴場飲食店への推薦のように,ユーザに新たな発見を促すような推薦は,セレンディピティが高いといわれる.推薦におけるセレンディピティとは,「思いがけないものを推薦すること」であり\cite{McNee},セレンディピティが高い推薦によって,他の方法では発見できない意外なものを推薦できる\cite{Parameswaran}.\par
加藤ら\cite{onomatopoeia}は,飲食店の口コミに含まれるオノマトペを用いてセレンディピティのある推薦方法を提案した.食感や見た目などをの要素を表すオノマトペから飲食店の類似度を計算し,ユーザに対して重視する要素を選ばせて推薦を行う.しかし,この方法では,口コミにオノマトペを含まない飲食店を推薦できない.知名度が低い飲食店は口コミの数が少ないと予想され,そういった飲食店を推薦するためには新たな手法が必要となる.
西脇ら\cite{picture}は,写真共有サイト内における,あるスポットのお気に入り数と撮影者の数を用いて,穴場撮影スポットを発見する方法を提案した.本研究はこの研究を飲食店推薦に応用する.\par
% \section{ユーザの行動履歴の利用に関連する研究}
% ユーザの行動履歴をもとに,飲食店が穴場である可能性を調べた例はない.
