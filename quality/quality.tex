\chapter{穴場飲食店の定義}

\label{chap:quality}

\section{穴場飲食店の定義}
本研究における穴場飲食店を「良質さが高く,かつ知名度が低い飲食店」と定める.
したがって,飲食店の穴場具合はその飲食店の良質さと知名度の二つの要素によって決まる.
そのため,本研究における良質さを明確に定める必要がある.
% そのため,本研究における良質さと知名度を定める必要がある.

\section{飲食店の良質さ}
飲食店の良質さは,TPOに応じて様々な要因により評価される.例えば,接待をするための飲食店を選ぶ場合と,平日の昼食をとるための飲食店を選ぶ場合とでは,良質さの評価基準や性質が異なるのは明らかである.よって,飲食店を評価する際には,目的に応じた適切に良質さを定めなければならない.
本研究では,穴場飲食店に求める良質さを「何度も繰り返し行きたくなる性質」と定める.
また,ホットペッパーグルメ外食総研の実施したアンケートによると,飲食店への来店のうち,リピート利用の占める割合は77.3\%であり,
リピート利用をする飲食店において,69.6\%の人が「料理がおいしい」という点を重視し,48.6\%の人が「コストパフォーマンスがよい」という点を重視している.
よって,本研究における飲食店の良質さは,料理のおいしさとコストパフォーマンスの二つによって決まるとする.

% \section{飲食店の地名度}
% 飲食店の知名度は,誰にとっての知名度なのかを設定しなければならない.
% 本研究における飲食店の知名度を,地元住民にどれぐらい知られているかを表すものと定める.
