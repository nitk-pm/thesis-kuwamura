\chapter{指標の提案}
\label{chap:proposal}

\section{概要}
新規来店者が少ない飲食店,ごく少数の熱心なリピーターがいるが延べ来店者数が少ない飲食店,および,多数のユーザが何度も来店しているが延べ来店者数は少ない店は穴場である可能性が高い.よって,これらを高く評価できる指標を提案する.\par
提案する指標は,来店者新規度,やみつき度,万人受け度の3つである.来店者新規度は新規来店者の割合を示す.やみつき度は熱心なリピーターがいる場合に大きい値をとる.万人受け度は何度も通うユーザが複数人いる場合に大きい値をとる.
%\section{指標に対する要件}

%推薦したいと考える穴場飲食店とは,推薦されにくいほどに知名度が低いが,推薦すべきほどに飲食店として良質であるという二つの性質を満たすものである.
%すなわち,ごく一部のユーザだけが何度も通っていて,その他のユーザはその存在に気が付いていない飲食店が穴場であると考える.
%そこで,一定期間の全ユーザの来店履歴から得られる,飲食店への来店者数の多さを知名度,のべ来店者数の多さを良質さとして,知名度が小さく,良質さが大きい飲食店を推薦すべきだという考えのもとに指標を提案する.

\section{来店者新規度(Visitor Novelty)}

来店者新規度は,ある飲食店に対する来店者数を延べ来店者数で割った値である.来店者新規度が高い飲食店は観光客がたくさん訪れる有名店,または一度行った人がもう行かないと判断した飲食店である.よって,来店者新規度が低いほど穴場であると考えられる.
ある飲食店に対する来店者数を$V$,延べ来店者数を$T$とすると,来店者新規度$\rm R$は式(\ref{equ:VN})で与えられる.ただし,同じユーザが$N$回来店している場合には,来店者数は1人と数え,延べ来店者数は$N$人と数える.
\begin{equation}
	\label{equ:VN}
	{\rm R} = \frac{V}{T}
\end{equation}

来店者新規度では,
\begin{itemize}
\item ある人が5回,別のある人が1回だけ来店した飲食店A
\item ある人が3回,別のある人が3回だけ来店した飲食店B
\end{itemize}
このような飲食店A,Bについての評価が同じだが,これらは明らかに違う性質を持っている.よって,これらを区別可能な指標を考案した.

\section{やみつき度(Addictivity)}

やみつき度は,全飲食店の延べ来店者数の平均を底,1回以上来店したユーザの来店回数を指数とした重みを持ち,
%全飲食店の延べ来店者数の平均を$(その飲食店に来店した各ユーザの来店回数-1)$乗した値について全ユーザの総和を取り延べ来店者数で割った値に$1$を加え,常用対数をとったものである.
来店頻度が高いユーザがいるかどうかを表している.よって,やみつき度が高いほど穴場であると考えられる.
%指数使う時点で単位が崩壊しており,平均取った時点で底の意味すら崩壊しているので,この値に数学的な意味はなく,説明は不可能です.
ある飲食店に来店したユーザの数を$N$,$i$番目の来店ユーザの来店回数を$V_i$,延べ来店者数を$T$,全飲食店の延べ来店者数の平均を$T_{ave}$とすると,やみつき度$\rm Y$は式(\ref{equ:addictivity})で与えられる.
\begin{equation}
	\label{equ:addictivity}
	{\rm Y} = \log_{10}{\left(\sum^{N}_{i=1} \frac{(T_{ave})^{V_i-1}-1}{T}+1\right)}
\end{equation}
やみつき度では飲食店Aのような傾向の飲食店を高く評価できる.
%これは,何度も来店する一人のユーザによって値が大きく左右されるため,飲食店Aのような傾向でユーザが来店する飲食店を高く評価できる.

\section{万人受け度(Acceptability)}

万人受け度は,ある飲食店に1回以上来店したユーザの来店回数の総積を延べ来店者数で割った数であり,複数のユーザが複数回来店しているが,延べ来店者数は少ないときに大きい値をとる.よって,万人受け度が高いほど穴場であると考えられる.
ある飲食店に来店したユーザの数を$N$,$i$番目の来店ユーザの来店回数を$V_i$,延べ来店者数を$T$とすると,万人受け度$\rm B$は式(\ref{equ:acceptability})で与えられる.
%単位崩壊2
\begin{equation}
	\label{equ:acceptability}
	{\rm B} = \frac{1}{T} \left(\prod^N_{i=1}V_i\right)
\end{equation}
万人受け度では飲食店Bのような傾向の飲食店を高く評価できる.
%これは,複数のユーザが複数回来店しているかどうかに値が大きく左右されるため,飲食店Bのような傾向でユーザが来店する飲食店を高く評価できる.

%人気ランキング上位の店について,指標上でのランキングをみる
