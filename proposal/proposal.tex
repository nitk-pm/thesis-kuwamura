\chapter{指標の提案}
\label{chap:proposal}

\section{概要}

飲食店推薦システムでは,知名度が高い飲食店は必然的に評価される機会が多く,推薦が容易であるのに対し,知名度の低い飲食店は推薦の根拠となる評価が少ないため推薦されにくいといった問題が考えられる.ここでは,全ユーザの来店履歴データをもとに,知名度は低いが良質な飲食店を推薦するための「飲食店の穴場具合」を表す新たな指標を提案する.

\section{指標に対する要件}

推薦したいと考える穴場飲食店とは,推薦されにくいほどに知名度が低いが,推薦すべきほどに飲食店として良質であるという二つの性質を満たすものである.
すなわち,ごく一部のユーザだけが何度も通っていて,その他のユーザはその存在に気が付いていない飲食店が穴場であると考える.
そこで,一定期間の全ユーザの来店履歴から得られる,飲食店への来店者数の多さを知名度,のべ来店者数の多さを良質さとして,知名度が小さく,良質さが大きい飲食店を推薦すべきだという考えのもとに指標を提案する.

\section{来店者新規度(Visitor Novelty)}

ここで提案する指標のひとつは,ある飲食店に対するのべ来店者数のうち来店回数が1回のユーザが占める割合であり,これを来店者新規度と名付ける.来店者新規度が高い飲食店は観光客がたくさん訪れる有名店,新規にオープンした店舗,または知名度が上昇している店舗であると考えられる.
ユーザ数を$N$,ある飲食店に対する来店者数を$V$,のべ来店者数を$T$とすると,来店者新規度$\rm VN$は次式で与えられる.
\begin{equation}
	{\rm VN} = \frac{V}{T}
\end{equation}

来店者新規度では,\\\\
飲食店A:ある人が5回,別のある人が1回だけ来店した飲食店\\
飲食店B:ある人が3回,別のある人が3回だけ来店した飲食店\\\\
この二つの飲食店についての評価値が同じであるが,これらは明らかに違う性質を持っていると考えられる.よって,これらを区別可能な二つの指標を提案する.

\section{やみつき度}

ある飲食店に来店したユーザの数を$N$,$i$番目の来店ユーザの来店回数を$V_i$,のべ来店者数を$T$,全飲食店ののべ来店者数の平均を$T_{ave}$とすると,病みつき度$\rm E$は次式で与えられる.
\begin{equation}
	{\rm E} = \log_{10}{\left(\sum^{N}_{i=1} \frac{(T_{ave})^{V_i-1}-1}{T}+1\right)}
\end{equation}
これは,何度も来店する一人のユーザによって値が大きく左右されるため,飲食店Aのような傾向でユーザが来店する飲食店を高く評価できる.

\section{万人受け度}
ある飲食店に来店したユーザの数を$N$,$i$番目の来店ユーザの来店回数を$V_i$,のべ来店者数を$T$とすると,万人受け度$\rm L$は次式で与えられる.
\begin{equation}
	{\rm L} = \prod^N_{i=1}\frac{V_i}{T}
\end{equation}
これは,複数のユーザが複数回来店しているかどうかに値が大きく左右されるため,飲食店Bのような傾向でユーザが来店する飲食店を高く評価できる.

%人気ランキング上位の店について,指標上でのランキングをみる
