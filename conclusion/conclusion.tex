\chapter{結言}
\label{chap:conclusion}

% 本論文では,「飲食店リピート実態&リピート要因調査」の結果から,飲食店の良質さ,および穴場飲食店とはなにかを定義し,飲食店の穴場具合を表す指標である「来店者新規度」「やみつき度」「万人受け度」を提案した.\par
 本論文では,知名度が低く,かつ良質な飲食店への推薦を実現するために,飲食店が穴場である可能性を表す指標である「来店者新規度」,「やみつき度」,および,「万人受け度」を提案した.\par
アンケートの結果により,来店者新規度がもっとも穴場を推薦できていることがわかった.また,万人受け度は穴場ではないが,良質な飲食店を推薦できていることがわかった.さらに,各指標により推薦された飲食店の業種の多様性から,提案指標は飲食店の業種の差に強いことがわかった.\par
% 今回は高松市内におけるSwarm連携ツイートから得られた来店履歴を用いて指標の評価を行ったが,他の地域や別のアプリケーションを用いて得られた来店履歴においても,提案指標が実際に有効であるのかについて検証を行う必要がある.\par
今回は高松市内におけるSwarmのチェックイン連携ツイートから得られた来店履歴を用いて指標の評価を行ったが,今後の課題として,高松市民から無作為に選んでアンケートを行い評価することや,他の地域や別のアプリケーションを用いて得られた来店履歴においても提案指標が実際に有効であるかどうかについて検証を行うことが挙げられる.\par
% また,アンケートにおける飲食店の料理のクオリティの項目は,全ての推薦方法で同程度に高評価であった.
% やみつき度と来店者新規度の推薦結果は重複が5軒あったが,その他の推薦結果を比較すると,重複は少ないといえる.
% よって,この評価の偏りは提示した画像から判断したことに起因する可能性がある.
% したがって,評価実験のうち,料理のクオリティを調べるにあたっては,実際にその店へ行くなど,画像の提示以外の方法を取り,指標の再評価をする必要がある.
