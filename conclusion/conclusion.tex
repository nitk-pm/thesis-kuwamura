\chapter{結言}
\label{chap:conclusion}

本論文では,「飲食店リピート実態&リピート要因調査」の結果から,飲食店の良質さ,および穴場飲食店とはなにかを定義し,飲食店の穴場具合を表す指標である「来店者新規度」「やみつき度」「万人受け度」を提案した.\par
食べログ上での指標値の比較やアンケートの結果により,来店者新規度がもっとも穴場を推薦できていることが分かった.また,万人受け度は穴場ではないが,良質な飲食店を推薦できていることが分かった.さらに,推薦された飲食店の業態の多様性から,提案指標は飲食店の業態の差に強いことが分かった.\par
今回は高松市内におけるSwarm連携ツイートから得られた来店履歴を用いて指標の評価を行ったが,他の地域や別のアプリケーションを用いて得られた来店履歴においても,提案指標が実際に有効であるのかについて検証を行う必要がある.\par
%また,「飲食店リピート実態&リピート要因調査」における目的型来店とその他の来店を区別し,目的型来店のみを集めた来店履歴を用いて指標を再評価するべきである.
また,アンケートにおける飲食店の料理のクオリティの項目は,全ての推薦方法で同程度に高評価であった.
やみつき度と来店者新規度の推薦結果は重複が5軒あったが,その他の推薦結果を比較すると,重複は少ないといえる.
よって,この評価の偏りは提示した画像から判断したことに起因する可能性がある.
したがって,評価実験のうち,料理のクオリティを調べるにあたっては,実際にその店へ行くなど,画像の提示以外の方法を取る必要がある.
