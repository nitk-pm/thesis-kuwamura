\chapter{考察}

%応用・利用,複合,推薦結果の考察,食べログとの差,結局穴場をどうやって評価するのか
\label{chap:discussion}

\section{各指標の推薦結果に関する考察}
提案する各指標が推薦する飲食店の料理のクオリティとコストパフォーマンスは,全体的に高評価であり,良質ではない店はほとんど推薦されていないと考えられる.
また,食べログ上の点数による推薦では,推薦結果が全てうどん屋であるのに対し,提案する指標では様々な業種の飲食店を推薦できている.
よって,提案する指標は飲食店の業種による評価基準の差に強いと考えられる.
なお,食べログ上の点数による推薦結果の偏りは,喜多方市のラーメンや宇都宮市の餃子など,安価な名物料理がある地域で散見される.
\par
\subsection{来店者新規度について}
来店者新規度による推薦結果については,アンケートで有名店であるという評価を受けたのが1軒だけであり,来店者新規度では有名店が非常に推薦されにくいことがわかる.
よって,実際の推薦システムにおいては,来店者新規度が一定以上の飲食店を推薦候補から外した上で,点数などの既存指標に基づいた推薦することで,穴場を推薦することができると考えられる.\par

\subsection{やみつき度について}
やみつき度による推薦結果については,来店者新規度の推薦結果と同じ飲食店が多いが,アンケート結果によると来店者新規度による推薦結果より知名度が高く,2020年1月22日時点で,高松市内でもっとも食べログ上の口コミ数が多い超有名店を含んでいる.
よって,穴場推薦のためにはやみつき度より来店者新規度のほうが適切である.\par

\subsection{万人受け度について}
万人受け度では,牛丼全国チェーン店が2軒推薦されており,全体的に知名度が高いと評価されている.
よって,推薦している飲食店は穴場とは言えないが,コストパフォーマンスは高く評価されており.推薦する飲食店はもっとも良質であるといえる.
よって,万人受け度は,穴場に注目した推薦ではなく,一般的な飲食店推薦の指標としてそのまま用いるべきであると考える.

%いらない気がしてきた
% \section{コストパフォーマンスに関する考察}
% 今回,食べログ上の点数による推薦結果は,全体的にコストパフォーマンスが低いと評価されている.
% これはアンケート対象のうどんに対するコストパフォーマンスのハードルがかなり高くなっていることに起因し,アンケート対象を香川県民以外から選ぶことによって大きく変わると考えられる.
% よって,香川県民にとってのうどんのような相場が特殊な飲食店は,出身地からレビューや評価を層化抽出する必要があると考えられる.
