%#!pdfpLaTeX
%
% 北村研究室用卒業論文・特別論文のTeXテンプレートファイル
% 本ファイルは非公式であり,表紙とアブストに関しては下記で公開されているワードの
% テンプレートを利用して作成したものが公式であるので,表紙とアブストはPDFにして
% 差し替えること.
% https://www.kagawa-nct.ac.jp/EE/local/index.html (学内限定アクセス)
%
% 2020年1月17日 北村大地作成
%

%%%%%%%%%%%%%%%%%%%%%%%%%%% 論文情報 %%%%%%%%%%%%%%%%%%%%%%%%%%%
%%%%% テンプレート選択 %%%%%
\documentclass[honka]{nitkcthesis}%卒論(本科5年)日本語用

\usepackage{here}
\usepackage{siunitx}

%%%%% タイトル %%%%%
\title{\underline{全ユーザの来店履歴を用いた}\\\underline{飲食店推薦指標の提案}}
%\titlewidth{}% タイトル幅 (指定するときは単位つきで)

%%%%% 著者 %%%%%
\author{name}
\eauthor{name}% Copyright表示で使われる

%%%%% 指導教員名 %%%%%
\supervisor{name}% 1つ引数をとる (役職まで含めて書く)


%%%%% 提出年月 %%%%%
\date{令和3年 3月 31日}

%%%%% \usepackage等のプリアンブル宣言(macros.texに記載) %%%%%
\usepackage{bm}
\usepackage{amsmath, amssymb}
\usepackage[dvipdfmx]{color}
\usepackage[dvipdfmx]{graphicx}
\usepackage{tabularx}
\usepackage{booktabs}
\usepackage{multirow}
\usepackage{setspace}
\usepackage{amsthm}
\usepackage[caption=false]{subfig}
\usepackage[numbers,sort]{natbib}

\theoremstyle{definition}
\newtheorem{theo}{定理}[chapter]
\newtheorem{defi}{定義}[chapter]
\newtheorem{lemm}{補題}[chapter]
\renewcommand{\proofname}{\textbf{証明}}

%% definition
\newcommand{\J}{\mathrm{j}}
\newcommand{\diag}{\mathop{\mathrm{diag}}}

\newcommand{\mtr}[1]{#1^{\mathsf{T}}}
\newcommand{\ctr}[1]{#1^{\mathsf{H}}}
\newcommand{\inv}[1]{#1^{-1}}
\newcommand{\cinv}[1]{#1^{-\mathsf{H}}}
\newcommand{\tinv}[1]{#1^{-\mathsf{T}}}
\newcommand{\conj}[1]{#1^*}

\newcommand{\tbm}[1]{\tilde{\bm{#1}}}
\newcommand{\tsf}[1]{\tilde{\mathsf{#1}}}

\newcommand{\vw}{\bm{w}}
\newcommand{\mW}{\bm{W}}
\newcommand{\vwhat}{\widehat{\bm{w}}}
\newcommand{\mWhat}{\widehat{\bm{W}}}

\newcommand{\hhat}{\widehat{h}}
\newcommand{\rhat}{\widehat{r}}

\newcommand{\argmax}{\mathop{\mathrm{arg~max}}\limits}
\newcommand{\argmin}{\mathop{\mathrm{arg~min}}\limits}

\renewcommand{\Re}{\mathop{\mathrm{Re}}}
\renewcommand{\Im}{\mathop{\mathrm{Im}}}

\newcommand{\unit}[1]{~\mathrm{#1}}
\newcommand{\Unit}[1]{~\mathrm{\left[#1\right]}}

\renewcommand{\qedsymbol}{$\blacksquare$}

\bibliographystyle{IEEEtran}

\makeatletter
\def\bstctlcite{\@ifnextchar[{\@bstctlcite}{\@bstctlcite[@auxout]}}
\def\@bstctlcite[#1]#2{\@bsphack
\@for\@citeb:=#2\do{%
\edef\@citeb{\expandafter\@firstofone\@citeb}%
\if@filesw\immediate\write\csname #1\endcsname{\string\citation{\@citeb}}\fi}%
\@esphack}
\makeatother


\begin{document}
\bstctlcite{IEEEexample:BSTcontrol} % BibTeXのIEEEtranで同一著者の横線表示を防止

\maketitle% タイトル生成

%%%%%%%%%%%%%%%%%%%%%%%%%%% 前文 %%%%%%%%%%%%%%%%%%%%%%%%%%%
\frontmatter

%%%%% English title %%%%%
\etitle{proposal of metrics for restaurant recommendation using all users' visitation history}

%%%%% Abstract %%%%%
\eabstract{
Evaluations in restaurant review sites are referenced for searching places to dine. In conventional review sites, little-known restaurants have low rates. However, little-known restaurants can still be high-quality despite being newly opened or lacking  advertising. Therefore, we propose three metrics that indicate cuisine quality and obscurity; ``Visitor Novelty,''  ``Addictivity,'' and ``Acceptability.'' To evaluate the metrics, we have compiled Swarm-associated tweets posted from 2020/07/28 to 2020/12/24 and calculated the indices of metrics. Based on the indices, we have compared the top 10 restaurants of the metrics-based recommendations with the top 10 restaurants from a conventional review site through a questionnaire to 8 male students. Compared to the ranking of the conventional review site, in our recommendations based on Visitor Novelty, 2 more restaurants were rated as ``not well-known,'' 2 fewer restaurants were rated as  ``would like to go,'' and 3 more restaurants were rated as ``not well-known and would like to go.'' Compared to the ranking of the conventional review site, in our recommendations based on Addictivity, 4 more restaurants were rated as ``not well-known,'' 2 fewer restaurants were rated as  ``would like to go,'' and 5 more restaurants were rated as ``not well-known and would like to go.'' Compared to the ranking of the conventional review site, in our recommendations based on Acceptability, 2 more restaurants were rated as ``not well-known,'' 2 fewer restaurants were rated as  ``would like to go,'' and 1 more restaurant was rated as ``not well-known and would like to go.'' According to the results, the metrics are more likely  to recommend hole-in-the-wall restaurants than the conventional review site. And it is thought that Addictivity is the best metric to recommend hole-in-the-walls.
%FIXME 数字はまだ妄想
}

%%%%% 概要 %%%%%
\abstract{
飲食店レビューサイト内の評価は飲食店を調べる際に頻繁に参考にされる.既存のレビューサイトでは,知名度が低い飲食店は評価が低い傾向にある.しかし,新規にオープンした飲食店や積極的に集客を行っていない飲食店など,知名度が低い飲食店の中にも良質な穴場店があると考えられる.そこで,本研究では,穴場店に対する推薦を実現するために,飲食店の穴場具合を表す推薦指標「来店者新規度」「やみつき度」「万人受け度」を提案する.各指標では飲食店に対する来店者数,延べ来店者数,および各来店者ごとの来店回数をもとに飲食店の穴場具合を判断する.評価実験のために,2020年7月28日から2020年12月24日までの高松市におけるSwarm連携ツイートを収集し,得られた来店履歴を用いて高松市内の飲食店に対して指標値を計算した.計算した指標値をもとに,高松市内の飲食店について,提案する各指標に基づく推薦結果と,既存のレビューサイトの高松市におけるランキングのそれぞれ上位10軒の知名度と良質さを男子高専生8名に対するアンケートにより比較した.既存のレビューサイトの推薦結果と比較して,来店者新規度に基づく推薦では「よく知られていないと思う」と評価された店が4軒多く,「行きたいと思う」と評価された店が2軒少なく,「よく知られていないと思う」かつ「行きたいと思う」と評価された飲食店数が3軒多くなった.既存のレビューサイトの推薦結果と比較して,やみつき度に基づく推薦では「よく知られていないと思う」と評価された店が4軒多く,「行きたいと思う」と評価された店が2軒少なく,「よく知られていないと思う」かつ「行きたいと思う」と評価された飲食店数が5軒多くなった.既存のレビューサイトの推薦結果と比較して,万人受け度に基づく推薦では「よく知られていないと思う」と評価された店が2軒多く,「行きたいと思う」と評価された店が2軒少なく,「よく知られていないと思う」かつ「行きたいと思う」と評価された飲食店数が1軒多くなった.したがって,各新規指標は既存のレビューサイトよりも穴場を推薦できていると考えられる.また,提案する指標のうち,やみつき度がもっともよく穴場を推薦できていると考えられる.
%FIXME 数字はまだ妄想
}

\keywords{Restaurant recommendation, Hole-in-the-wall, Visitation history}

\makeseparatedabstract
%\makeabstract
%%%%% 目次 %%%%%
%\tableofcontents % ページ番号を削除しない目次
%----- ページ番号を削除した目次 -----%
{\makeatletter
\let\ps@jpl@in\ps@empty
\makeatother
\pagestyle{empty}
\tableofcontents
\clearpage}
%---------------------------------%

%%%%%%%%%%%%%%%%%%%%%%%%%%% 本文 %%%%%%%%%%%%%%%%%%%%%%%%%%%
\mainmatter

\chapter{緒言}
\label{chap:intro}
身近な飲食店推薦サービスである食べログ[N]やRetty[N]では,有志のユーザによるレビューをもとに飲食店の良し悪しを評価する,集合知による推薦が行われている.
しかし,この方式では,飲食店の質に関わらず,知名度が低い場合は高い点数を得られない.
特に食べログでは,評価の数が少ない場合,点数が低くなることが明言されている[N].
新規オープンした飲食店や熱心な集客活動を行っていない飲食店は,その質によらず,知名度や評価数の点で不利である.
したがって,知名度が高い店は推薦されやすく,多数の評価を得られるのに対し,知名度が低い店は推薦されにくく,なかなか評価を得ることができないという問題が生じる
.この問題を解消するため,良質だが知名度が低い穴場飲食店への推薦が求められている.

飲食店が穴場であるという根拠になるのが,少数のリピーターによる頻繁な来店である.
本研究では,穴場飲食店とは「少数のユーザが頻繁に通い,その他のユーザはその存在に気づいていない飲食店」であるという仮定のもと,来店履歴データを利用した指標を考案する.
また,本研究では,リクルートライフスタイルによる「飲食店リピート実態&リピート要因調査」[N]をもとに,本研究における「飲食店の良質さ」を定め,指標の評価に用いる.

さらに,近年では,スマートフォンの普及に伴い,多くの人々がSNS上で手軽に様々な情報を共有するようになった.
その中でも,位置情報を共有するSNSであるSwarm[N]は,スマートフォンのGPS機能を使って,自分が現在どこにいるかを「チェックイン」という形で記録,共有できる.
また,Twitter[N]と連携し,チェックイン情報をツイートすることで,ライフログを公開データとして記録することができる.
本研究では,Twitter上のSwarmのチェックインに連携したツイートをもとに来店履歴データを作成し,高松市内の飲食店に対して指標値を実際に計算する.
得られた指標値をもとに推薦された飲食店と既存サービスにより推薦された飲食店に対して,良質さと知名度をアンケートで比較し,指標が穴場を推薦できているかの評価を行う.


\chapter{穴場飲食店の定義}

\label{chap:quality}

\section{穴場飲食店の定義}
本研究における穴場飲食店を「良質さが高く,かつ知名度が低い飲食店」と定める.
したがって,飲食店の穴場具合はその飲食店の良質さと知名度の二つの要素によって決まる.
そのため,本研究における良質さを明確に定める必要がある.
% そのため,本研究における良質さと知名度を定める必要がある.

\section{飲食店の良質さ}
飲食店の良質さは,TPOに応じて様々な要因により評価される.例えば,接待をするための飲食店を選ぶ場合と,平日の昼食をとるための飲食店を選ぶ場合とでは,良質さの評価基準や性質が異なるのは明らかである.よって,飲食店を評価する際には,目的に応じた適切に良質さを定めなければならない.
本研究では,穴場飲食店に求める良質さを「何度も繰り返し行きたくなる性質」と定める.
また,ホットペッパーグルメ外食総研の実施したアンケートによると,飲食店への来店のうち,リピート利用の占める割合は77.3\%であり,
リピート利用をする飲食店において,69.6\%の人が「料理がおいしい」という点を重視し,48.6\%の人が「コストパフォーマンスがよい」という点を重視している.
よって,本研究における飲食店の良質さは,料理のおいしさとコストパフォーマンスの二つによって決まるとする.

% \section{飲食店の地名度}
% 飲食店の知名度は,誰にとっての知名度なのかを設定しなければならない.
% 本研究における飲食店の知名度を,地元住民にどれぐらい知られているかを表すものと定める.


\chapter{指標の提案}
\label{chap:proposal}

\section{概要}
新規来店者が少ない飲食店,ごく少数の熱心なリピーターがいるが延べ来店者数が少ない飲食店,および,多数のユーザが何度も来店しているが延べ来店者数は少ない店は穴場である可能性が高い.よって,これらを高く評価できる指標を提案する.\par
提案する指標は,来店者新規度,やみつき度,万人受け度の3つである.来店者新規度は新規来店者の割合を示す.やみつき度は熱心なリピーターがいる場合に大きい値をとる.万人受け度は何度も通うユーザが複数人いる場合に大きい値をとる.
%\section{指標に対する要件}

%推薦したいと考える穴場飲食店とは,推薦されにくいほどに知名度が低いが,推薦すべきほどに飲食店として良質であるという二つの性質を満たすものである.
%すなわち,ごく一部のユーザだけが何度も通っていて,その他のユーザはその存在に気が付いていない飲食店が穴場であると考える.
%そこで,一定期間の全ユーザの来店履歴から得られる,飲食店への来店者数の多さを知名度,のべ来店者数の多さを良質さとして,知名度が小さく,良質さが大きい飲食店を推薦すべきだという考えのもとに指標を提案する.

\section{来店者新規度(Visitor Novelty)}

来店者新規度は,ある飲食店に対する来店者数を延べ来店者数で割った値である.来店者新規度が高い飲食店は観光客がたくさん訪れる有名店,または一度行った人がもう行かないと判断した飲食店である.よって,来店者新規度が低いほど穴場であると考えられる.
ユーザ数を$N$,ある飲食店に対する来店者数を$V$,延べ来店者数を$T$とすると,来店者新規度$\rm R$は式(\ref{equ:VN})で与えられる.
\begin{equation}
	\label{equ:VN}
	{\rm R} = \frac{V}{T}
\end{equation}

% 来店者新規度では,\\\\
% 飲食店A:ある人が5回,別のある人が1回だけ来店した飲食店\\
% 飲食店B:ある人が3回,別のある人が3回だけ来店した飲食店\\\\
% このひとつの飲食店についての評価値が同じであるが,これらは明らかに違う性質を持っていると考えられる.よって,これらを区別可能なひとつの指標を提案する.

\section{やみつき度(Addictivity)}

やみつき度は,全飲食店の延べ来店者数の平均を底,1回以上来店したユーザの来店回数を指数とした重みを持ち,
%全飲食店の延べ来店者数の平均を$(その飲食店に来店した各ユーザの来店回数-1)$乗した値について全ユーザの総和を取り延べ来店者数で割った値に$1$を加え,常用対数をとったものである.
来店頻度が高いユーザがいるかどうかを表している.
%指数使う時点で単位が崩壊しており,平均取った時点で底の意味すら崩壊しているので,この値に数学的な意味はなく,説明は不可能です.
ある飲食店に来店したユーザの数を$N$,$i$番目の来店ユーザの来店回数を$V_i$,延べ来店者数を$T$,全飲食店の延べ来店者数の平均を$T_{ave}$とすると,やみつき度$\rm Y$は式(\ref{equ:addictivity})で与えられる.
\begin{equation}
	\label{equ:addictivity}
	{\rm Y} = \log_{10}{\left(\sum^{N}_{i=1} \frac{(T_{ave})^{V_i-1}-1}{T}+1\right)}
\end{equation}
%これは,何度も来店する一人のユーザによって値が大きく左右されるため,飲食店Aのような傾向でユーザが来店する飲食店を高く評価できる.

\section{万人受け度(Acceptability)}

万人受け度は,ある飲食店に1回以上来店したユーザの来店回数の総積を延べ来店者数で割った数であり,複数のユーザが複数回来店しているかどうかによってとる値が大きく変わる.
ある飲食店に来店したユーザの数を$N$,$i$番目の来店ユーザの来店回数を$V_i$,延べ来店者数を$T$とすると,万人受け度$\rm B$は式(\ref{equ:acceptability})で与えられる.
%単位崩壊2
\begin{equation}
	\label{equ:acceptability}
	{\rm B} = \frac{1}{T} \left(\prod^N_{i=1}V_i\right)
\end{equation}
%これは,複数のユーザが複数回来店しているかどうかに値が大きく左右されるため,飲食店Bのような傾向でユーザが来店する飲食店を高く評価できる.

%人気ランキング上位の店について,指標上でのランキングをみる


\chapter{指標の評価}


% \newcommand{\ctext}[1]{\textcircled{\scriptsize #1}}
%丸囲み数字

\label{chap:experiment}

\section{目的}
提案する指標が,既存のサービスの推薦と比べて,より穴場である可能性が高い店を推薦することができているかを確認するため,評価を行った.

\section{使用する来店履歴データ}
高松市内の飲食店に対しての指標値を計算するため,
高松市における2020年7月28日から2020年12月24日までに呟かれた位置情報共有サービスSwarmのチェックインに連携したツイートをTwitter社の提供するSearch APIを用いて収集した.収集した10055件のチェックイン連携ツイートのうち,飲食店へのチェックインである2287件から得られた来店履歴を用いて高松市内の飲食店に対して各指標値を計算した.チェックイン先が飲食店であるかどうかは,インターネット検索により店舗が飲食店に該当するかを調べて判断した.ただし,飲食店とは,日本標準産業分類による定義\cite{restaurant}に従い,「主として注文により直ちにその場所で料理、その他の食料品または飲料を飲食させる事業所」を指す.テイクアウト専門店やスーパーマーケットなどは食料品を提供するが,本実験においては飲食店とは見なさない.また,複数の飲食店により構成されるフードコートなどの施設は飲食店とは見なさない.\par
得られた来店履歴データより,高松市において,
飲食店へチェックインしたユーザは842人,
チェックインされた飲食店は597軒,
チェックインされた全飲食店の平均延べ来店者数は3.85人であった.

\section{評価方法}
提案する各指標値が高い飲食店上位10軒,および既存の飲食店推薦サービスである食べログ上の点数ランキング機能における2021年1月19日時点の高松市のランキング上位10軒からなる飲食店群に対して,後述するアンケートを用いて,飲食店の良質さと知名度を調査した.
ただし,高級店に日常的に通うとは考えにくく,提案する指標の利用目的に合致しないことが明らかであるため,食べログ上の,ランチとディナーの予算のうち安価な方が5000円を超えるような飲食店は推薦結果から除外したうえで,各推薦方法による上位10軒の飲食店を選んだ.
	% \subsection{推薦された飲食店の食べログ上の指標の調査}\label{exp:scrutiny}
    %
	% 	上述の飲食店群に含まれる飲食店に対して食べログ上の指標値である,点数と口コミの数とブックマークの数を調べた.また,推薦基準ごとに食べログ上の指標値の平均を求めた.

% \subsection{アンケート内容}\label{exp:questionnaire}
\section{アンケート内容}\label{exp:questionnaire}

		高松市在住の男子高専生8人に対して
		\begin{itemize}
			\item 今まで行ったことがない飲食店に積極的に行ってみたいと思うか
			\item 穴場に注目した飲食店推薦サービスがあったら利用したいと思うか
			\item 外食の頻度
		\end{itemize}
		提案指標,および食べログ上のランキングによって推薦された各飲食店について,
		\begin{itemize}
			\item 飲食店の知名度
			\item 飲食店の料理のクオリティ
			\item 飲食店のコストパフォーマンス
			\item 飲食店に行きたいと思うか
		\end{itemize}
		がアンケート用Webページを配布し,以下の項目について質問した.アンケート結果は,xlsxファイルに記入させて回収した.
		% ただし,本実験では,飲食店の良質さが料理のクオリティ,およびコストパフォーマンスのふたつの要素から決定されると仮定した.\par
% \section{結果}
	% \subsection{食べログ上の指標値の調査結果}
	% 各指標の推薦結果上位10軒,および食べログの点数ランキング上位10軒に対して,各指標値,2020年1月22日時点における食べログ上の点数,2020年1月22日時点における食べログ上の口コミ数,2020年1月22日時点における食べログ上のブックマーク数について調べ,表\ref{table:scrutiny:VN}〜\ref{table:scrutiny:rank}にまとめた.表内の飲食店は推薦順位が高い順に並べている.また,食べログ上の点数,食べログ上の口コミ数,食べログ上のブックマーク数について,推薦方法ごとに平均値を計算し,表\ref{table:scrutiny:average}にまとめた.
	% % Please add the following required packages to your document preamble:
% \usepackage{multirow}
\begin{table}[H]
\centering
\caption{来店者新規度による推薦結果上位10軒}
\label{table:scrutiny:VN}
\small
\begin{tabular}{|c|r|r|r|r|r|r|}
\hline
\multirow{2}{*}{店名} & \multicolumn{3}{c|}{提案指標} & \multicolumn{3}{c|}{食べログ上の指標} \\ \cline{2-7}
 & \multicolumn{1}{c|}{来店者新規度} & \multicolumn{1}{c|}{やみつき度} & \multicolumn{1}{c|}{万人受け度} & \multicolumn{1}{c|}{点数} & \multicolumn{1}{c|}{口コミ数} & \multicolumn{1}{c|}{ブックマーク数} \\ \hline
炭火焼鳥 満天 & 0.143 & 3.513 & 1.000 & 3.07 & 4 & 79 \\ \hline
上海小籠包 & 0.200 & 2.342 & 1.000 & なし & なし & なし \\ \hline
つばめ家 & 0.200 & 2.342 & 1.000 & 3.30 & 11 & 427 \\ \hline
炭焼やき鳥 三吉 & 0.250 & 1.756 & 1.000 & 3.03 & 1 & 40 \\ \hline
武内食堂 鍛冶屋町店 & 0.250 & 1.756 & 1.000 & 3.08 & 4 & 94 \\ \hline
ヨコクラうどん & 0.250 & 3.212 & 0.875 & 3.62 & 54 & 1339 \\ \hline
ときわ食堂 & 0.250 & 1.756 & 1.000 & 3.00 & 4 & 38 \\ \hline
うどん市場 めんくい & 0.250 & 1.756 & 1.000 & 3.51 & 26 & 927 \\ \hline
Café Buono & 0.250 & 1.756 & 1.000 & 3.27 & 31 & 290 \\ \hline
Amazon & 0.250 & 4.792 & 0.833 & 3.15 & 6 & 96 \\ \hline
\end{tabular}
\end{table}

	% % Please add the following required packages to your document preamble:
% \usepackage{multirow}
\begin{table}[H]
\centering
\caption{やみつき度による推薦結果上位10軒}
\label{table:scrutiny:addictivity}
\small
\begin{tabular}{|c|r|r|r|r|r|r|}
\hline
\multirow{2}{*}{店名} & \multicolumn{3}{c|}{提案指標} & \multicolumn{3}{c|}{食べログ上の指標} \\ \cline{2-7}
 & \multicolumn{1}{c|}{来店者新規度} & \multicolumn{1}{c|}{やみつき度} & \multicolumn{1}{c|}{万人受け度} & \multicolumn{1}{c|}{点数} & \multicolumn{1}{c|}{口コミ数} & \multicolumn{1}{c|}{ブックマーク数} \\ \hline
ホープ軒 福岡店 & 0.333 & 5.156 & 0.733 & 3.36 & 45 & 840 \\ \hline
手打十段 うどんバカ一代 & 0.890 & 4.887 & 2.684 & 3.79 & 1034 & 38204 \\ \hline
Amazon & 0.250 & 4.792 & 0.833 & 3.15 & 6 & 96 \\ \hline
OTTIMO イオンモール高松 & 0.560 & 3.538 & 2.880 & 3.04 & 3 & 24 \\ \hline
炭火焼鳥 満天 & 0.143 & 3.513 & 1.000 & 3.07 & 4 & 79 \\ \hline
ヨコクラうどん & 0.250 & 3.212 & 0.875 & 3.62 & 54 & 1339 \\ \hline
朔日 & 0.286 & 2.627 & 0.857 & 3.17 & 12 & 745 \\ \hline
手打ちうどん ますや & 0.375 & 2.451 & 0.750 & 3.67 & 44 & 2069 \\ \hline
上海小籠包 & 0.200 & 2.342 & 1.000 & なし & なし & なし \\ \hline
つばめ家 & 0.200 & 2.342 & 1.000 & 3.30 & 11 & 427 \\ \hline
\end{tabular}
\end{table}

	% % Please add the following required packages to your document preamble:
% \usepackage{multirow}
\begin{table}[H]
\centering
\caption{万人受け度による推薦結果上位10軒}
\label{table:scrutiny:acceptability}
\small
\begin{tabular}{|c|r|r|r|r|r|r|}
\hline
\multirow{2}{*}{店名} & \multicolumn{3}{c|}{提案指標} & \multicolumn{3}{c|}{食べログ上の指標} \\ \cline{2-7}
 & \multicolumn{1}{c|}{来店者新規度} & \multicolumn{1}{c|}{やみつき度} & \multicolumn{1}{c|}{万人受け度} & \multicolumn{1}{c|}{点数} & \multicolumn{1}{c|}{口コミ数} & \multicolumn{1}{c|}{ブックマーク数} \\ \hline
めりけんや 高松駅前店 & 0.842 & 1.187 & 3.789 & 3.49 & 287 & 7275 \\ \hline
OTTIMO イオンモール高松 & 0.560 & 3.538 & 2.880 & 3.04 & 3 & 24 \\ \hline
手打十段 うどんバカ一代 & 0.890 & 4.886 & 2.684 & 3.79 & 1034 & 38204 \\ \hline
すき家 高松寿町店 & 0.400 & 1.283 & 2.400 & なし & なし & なし \\ \hline
裏きせき & 0.529 & 1.503 & 2.353 & 3.37 & 21 & 686 \\ \hline
海鮮食堂 じゃこや & 0.563 & 1.139 & 2.000 & 3.51 & 76 & 2741 \\ \hline
ダントツラーメン 岡山一番店 & 0.615 & 0.941 & 1.231 & 3.28 & 34 & 696 \\ \hline
しんみょう精肉店 鍛冶屋町店 & 0.400 & 0.970 & 1.200 & 3.04 & 3 & 243 \\ \hline
吉野家 高松瓦町店 & 0.400 & 0.970 & 1.200 & なし & なし & なし \\ \hline
麺屋 がんてつ & 0.667 & 1.791 & 1.143 & 3.38 & 52 & 703 \\ \hline
\end{tabular}
\end{table}

	% % Please add the following required packages to your document preamble:
% \usepackage{multirow}
\begin{table}[H]
\centering
\caption{食べログ上の点数による推薦結果上位10軒}
\label{table:scrutiny:rank}
\small
\begin{tabular}{|c|S|S|S|S|S|S|}
\hline
\multirow{2}{*}{店名} & \multicolumn{3}{c|}{提案指標} & \multicolumn{3}{c|}{食べログ上の指標} \\ \cline{2-7}
 & \multicolumn{1}{c|}{来店者新規度} & \multicolumn{1}{c|}{やみつき度} & \multicolumn{1}{c|}{万人受け度} & \multicolumn{1}{c|}{点数} & \multicolumn{1}{c|}{口コミ数} & \multicolumn{1}{c|}{ブックマーク数} \\ \hline
瀬戸晴れ & 0.750 & 0.290 & 0.500 & 3.94 & 77 & 3685 \\ \hline
うどん 一福 & 0.882 & 0.140 & 0.235 & 3.91 & 493 & 26692 \\ \hline
手打うどん 麦蔵 & 1.000 & 0.000 & 0.333 & 3.89 & 200 & 16172 \\ \hline
手打ちうどん はりや & 1.000 & 0.000 & 0.143 & 3.87 & 178 & 8878 \\ \hline
手打ちうどん 大蔵 & 1.000 & 0.000 & 0.500 & 3.85 & 51 & 2540 \\ \hline
うどん本陣 山田家 讃岐本店 & 0.952 & 0.059 & 0.193 & 3.82 & 452 & 20482 \\ \hline
ふる里うどん & なし & なし & なし & 3.81 & 40 & 1551 \\ \hline
手打十段 うどんバカ一代 & 0.890 & 4.886 & 2.684 & 3.79 & 1034 & 38204 \\ \hline
さか枝 & 0.909 & 0.109 & 0.182 & 3.79 & 540 & 20382 \\ \hline
竹清 & 0.912 & 0.100 & 0.167 & 3.78 & 423 & 20455 \\ \hline
\end{tabular}
\end{table}

	% \begin{table}[H]
\centering
\caption{推薦基準ごとの食べログ上の指標の平均}
\label{table:scrutiny:average}
\small
\begin{tabular}{|c|S|S|S|}
\hline
推薦基準 & \multicolumn{1}{c|}{平均点数} & \multicolumn{1}{c|}{平均口コミ数} & \multicolumn{1}{c|}{平均ブックマーク数} \\ \hline
来店者新規度 & 3.226 & 15.7 & 370.0 \\ \hline
やみつき度 & 3.352 & 134.8 & 4869.2 \\ \hline
万人受け度 & 3.298 & 148.5 & 5068.6 \\ \hline
食べログ上のランキング & 3.845 & 348.8 & 15904.1 \\ \hline
\end{tabular}
\end{table}

	% \newpage

	% \subsection{アンケート結果}
	\section{アンケート結果}
	\ref{exp:questionnaire}に示した
	%各指標の推薦結果上位10軒,および食べログの点数ランキング上位10軒に対しての
	アンケートの結果を,表\ref{table:questionnaire:tendency}〜\ref{table:questionnaire:rank}にまとめた.表内の飲食店は推薦順位が高い順に並べている.また,各推薦方法,および各項目に対し「低い/絶対に行きたくない」または「やや低い/あまり行きたくない」と評価された数と「やや高い/少し行ってみたい」または「高い/かなり行ってみたい」と評価された数を表\ref{table:questionnaire:sum}にまとめた.
	\begin{table}[H]
\centering
\caption{外食の傾向についてのアンケート結果}
\label{table:questionnaire:tendency}
\small
\scalebox{0.7}[1.0]{
\begin{tabular}{|l|l|l|}
\hline
項目 & はい & いいえ \\ \hline
行ったことない店に行きたいか & 4 & 4 \\ \hline
穴場飲食店推薦サービスを利用したいか & 6 & 2 \\ \hline
\end{tabular}
}
\end{table}

	\begin{table}[H]
\centering
\caption{外食の傾向についてのアンケート結果}
\label{table:questionnaire:frequency}
\small
\begin{tabular}{|c|r|r|r|r|}
\hline
項目 & 年に数回 & 月に数回 & 週に数回 & ほぼ毎日 \\ \hline
外食の頻度 & 1 & 6 & 1 & 0 \\ \hline
\end{tabular}
\end{table}

	% Please add the following required packages to your document preamble:
% \usepackage{multirow}
\begin{table}[H]
\centering
\caption{来店者新規度による推薦のアンケート結果抜粋}
\label{table:questionnaire:VN}
\small
\scalebox{0.7}[1.0]{
\begin{tabular}{|l|l|l|l|l|l|}
\hline
店名 & 項目 & 低い/絶対に行きたくない & やや低い/あまり行きたくない & やや高い/少し行ってみたい & 高い/かなり行ってみたい \\ \hline
\multirow{4}{*}{炭火焼鳥 満天} & 知名度 & 5 & 1 & 1 & 1 \\ \cline{2-6}
 & 料理のクオリティ & 0 & 1 & 3 & 4 \\ \cline{2-6}
 & コストパフォーマンス & 0 & 4 & 4 & 0 \\ \cline{2-6}
 & 行きたいと思うか & 1 & 1 & 3 & 3 \\ \hline
\multirow{4}{*}{上海小籠包} & 知名度 & 4 & 2 & 2 & 0 \\ \cline{2-6}
 & 料理のクオリティ & 0 & 0 & 3 & 5 \\ \cline{2-6}
 & コストパフォーマンス & 0 & 0 & 5 & 3 \\ \cline{2-6}
 & 行きたいと思うか & 0 & 0 & 4 & 4 \\ \hline
\multirow{4}{*}{つばめ家} & 知名度 & 6 & 0 & 2 & 0 \\ \cline{2-6}
 & 料理のクオリティ & 0 & 1 & 3 & 4 \\ \cline{2-6}
 & コストパフォーマンス & 1 & 3 & 3 & 1 \\ \cline{2-6}
 & 行きたいと思うか & 0 & 4 & 3 & 1 \\ \hline
\end{tabular}
}
\end{table}

	% Please add the following required packages to your document preamble:
% \usepackage{multirow}
\begin{table}[H]
\centering
\caption{やみつき度による推薦のアンケート結果}
\label{table:questionnaire:addictivity}
\small
\begin{tabular}{|c|c|r|r|r|r|}
\hline
店名 & 項目 & \multicolumn{1}{Wc{4em}|}{低い} & \multicolumn{1}{Wc{4em}|}{やや低い} & \multicolumn{1}{Wc{4em}|}{やや高い} & \multicolumn{1}{Wc{4em}|}{高い} \\ \hline
\multirow{4}{*}{\begin{tabular}[c]{@{}c@{}}ホープ軒 \\ 福岡店\end{tabular}} & 知名度 & 3 & 1 & 2 & 2 \\ \cline{2-6} 
 & 料理のクオリティ & 0 & 1 & 4 & 3 \\ \cline{2-6} 
 & コストパフォーマンス & 1 & 3 & 3 & 1 \\ \cline{2-6} 
 & 行きたいと思うか & 0 & 2 & 4 & 2 \\ \hline
\multirow{4}{*}{\begin{tabular}[c]{@{}c@{}}手打十段 \\ うどんバカ一代\end{tabular}} & 知名度 & 3 & 1 & 1 & 3 \\ \cline{2-6} 
 & 料理のクオリティ & 0 & 0 & 5 & 3 \\ \cline{2-6} 
 & コストパフォーマンス & 0 & 1 & 7 & 0 \\ \cline{2-6} 
 & 行きたいと思うか & 0 & 1 & 5 & 2 \\ \hline
\multirow{4}{*}{Amazon} & 知名度 & 7 & 1 & 0 & 0 \\ \cline{2-6} 
 & 料理のクオリティ & 2 & 2 & 4 & 0 \\ \cline{2-6} 
 & コストパフォーマンス & 2 & 1 & 4 & 1 \\ \cline{2-6} 
 & 行きたいと思うか & 2 & 6 & 0 & 0 \\ \hline
\multirow{4}{*}{\begin{tabular}[c]{@{}c@{}}OTTIMO \\ イオンモール高松\end{tabular}} & 知名度 & 4 & 3 & 1 & 0 \\ \cline{2-6} 
 & 料理のクオリティ & 0 & 0 & 6 & 2 \\ \cline{2-6} 
 & コストパフォーマンス & 0 & 3 & 5 & 0 \\ \cline{2-6} 
 & 行きたいと思うか & 0 & 2 & 5 & 1 \\ \hline
\multirow{4}{*}{\begin{tabular}[c]{@{}c@{}}炭火焼鳥 \\ 満天\end{tabular}} & 知名度 & 5 & 1 & 1 & 1 \\ \cline{2-6} 
 & 料理のクオリティ & 0 & 1 & 3 & 4 \\ \cline{2-6} 
 & コストパフォーマンス & 0 & 4 & 4 & 0 \\ \cline{2-6} 
 & 行きたいと思うか & 1 & 1 & 3 & 3 \\ \hline
\multirow{4}{*}{ヨコクラうどん} & 知名度 & 5 & 1 & 2 & 0 \\ \cline{2-6} 
 & 料理のクオリティ & 0 & 1 & 6 & 1 \\ \cline{2-6} 
 & コストパフォーマンス & 0 & 1 & 4 & 3 \\ \cline{2-6} 
 & 行きたいと思うか & 0 & 2 & 4 & 2 \\ \hline
\multirow{4}{*}{朔日} & 知名度 & 7 & 0 & 1 & 0 \\ \cline{2-6} 
 & 料理のクオリティ & 0 & 1 & 3 & 4 \\ \cline{2-6} 
 & コストパフォーマンス & 1 & 2 & 4 & 1 \\ \cline{2-6} 
 & 行きたいと思うか & 0 & 1 & 6 & 1 \\ \hline
\multirow{4}{*}{\begin{tabular}[c]{@{}c@{}}手打ちうどん \\ ますや\end{tabular}} & 知名度 & 4 & 2 & 1 & 1 \\ \cline{2-6} 
 & 料理のクオリティ & 0 & 2 & 3 & 3 \\ \cline{2-6} 
 & コストパフォーマンス & 0 & 1 & 5 & 2 \\ \cline{2-6} 
 & 行きたいと思うか & 0 & 2 & 4 & 2 \\ \hline
\multirow{4}{*}{上海小籠包} & 知名度 & 4 & 2 & 2 & 0 \\ \cline{2-6} 
 & 料理のクオリティ & 0 & 0 & 3 & 5 \\ \cline{2-6} 
 & コストパフォーマンス & 0 & 0 & 5 & 3 \\ \cline{2-6} 
 & 行きたいと思うか & 0 & 0 & 4 & 4 \\ \hline
\multirow{4}{*}{つばめ家} & 知名度 & 6 & 0 & 2 & 0 \\ \cline{2-6} 
 & 料理のクオリティ & 0 & 1 & 3 & 4 \\ \cline{2-6} 
 & コストパフォーマンス & 1 & 3 & 3 & 1 \\ \cline{2-6} 
 & 行きたいと思うか & 0 & 4 & 3 & 1 \\ \hline
\end{tabular}
\end{table}

	% Please add the following required packages to your document preamble:
% \usepackage{multirow}
\begin{table}[H]
\centering
\caption{万人受け度による推薦のアンケート結果抜粋}
\label{table:questionnaire:acceptability}
\small
\scalebox{0.7}[1.0]{
\begin{tabular}{|l|l|l|l|l|l|}
\hline
店名 & 項目 & 低い/絶対に行きたくない & やや低い/あまり行きたくない & やや高い/少し行ってみたい & 高い/かなり行ってみたい \\ \hline
\multirow{4}{*}{めりけんや 高松駅前店} & 知名度 & 3 & 1 & 3 & 1 \\ \cline{2-6}
 & 料理のクオリティ & 0 & 3 & 4 & 1 \\ \cline{2-6}
 & コストパフォーマンス & 0 & 1 & 4 & 3 \\ \cline{2-6}
 & 行きたいと思うか & 0 & 3 & 4 & 1 \\ \hline
\multirow{4}{*}{OTTIMO イオンモール高松} & 知名度 & 4 & 3 & 1 & 0 \\ \cline{2-6}
 & 料理のクオリティ & 0 & 0 & 6 & 2 \\ \cline{2-6}
 & コストパフォーマンス & 0 & 3 & 5 & 0 \\ \cline{2-6}
 & 行きたいと思うか & 0 & 2 & 5 & 1 \\ \hline
\multirow{4}{*}{手打十段 うどんバカ一代} & 知名度 & 3 & 1 & 1 & 3 \\ \cline{2-6}
 & 料理のクオリティ & 0 & 0 & 5 & 3 \\ \cline{2-6}
 & コストパフォーマンス & 0 & 1 & 7 & 0 \\ \cline{2-6}
 & 行きたいと思うか & 0 & 1 & 5 & 2 \\ \hline
\end{tabular}
}
\end{table}

	% Please add the following required packages to your document preamble:
% \usepackage{multirow}
\begin{table}[H]
\centering
\caption{食べログ上の点数による推薦のアンケート結果}
\label{table:questionnaire:rank}
\small
\begin{tabular}{|c|c|r|r|r|r|}
\hline
店名 & 項目 & \multicolumn{1}{Wc{4em}|}{低い} & \multicolumn{1}{Wc{4em}|}{やや低い} & \multicolumn{1}{Wc{4em}|}{やや高い} & \multicolumn{1}{Wc{4em}|}{高い} \\ \hline
\multirow{4}{*}{おうどん 瀬戸晴れ} & 知名度 & 5 & 1 & 1 & 1 \\ \cline{2-6}
 & クオリティ & 0 & 0 & 2 & 6 \\ \cline{2-6}
 & コストパフォーマンス & 1 & 3 & 4 & 0 \\ \cline{2-6}
 & 行きたいと思うか & 0 & 2 & 3 & 3 \\ \hline
\multirow{4}{*}{うどん 一福} & 知名度 & 3 & 1 & 3 & 1 \\ \cline{2-6}
 & 料理のクオリティ & 0 & 1 & 5 & 2 \\ \cline{2-6}
 & コストパフォーマンス & 0 & 1 & 2 & 5 \\ \cline{2-6}
 & 行きたいと思うか & 0 & 3 & 4 & 1 \\ \hline
\multirow{4}{*}{手打うどん 麦蔵} & 知名度 & 5 & 1 & 2 & 0 \\ \cline{2-6}
 & 料理のクオリティ & 0 & 3 & 4 & 1 \\ \cline{2-6}
 & コストパフォーマンス & 2 & 5 & 1 & 0 \\ \cline{2-6}
 & 行きたいと思うか & 1 & 5 & 2 & 0 \\ \hline
\multirow{4}{*}{手打うどん はりや} & 知名度 & 4 & 1 & 3 & 0 \\ \cline{2-6}
 & 料理のクオリティ & 0 & 0 & 2 & 6 \\ \cline{2-6}
 & コストパフォーマンス & 3 & 0 & 4 & 1 \\ \cline{2-6}
 & 行きたいと思うか & 1 & 0 & 3 & 4 \\ \hline
\multirow{4}{*}{手打ちうどん大蔵} & 知名度 & 5 & 1 & 1 & 1 \\ \cline{2-6}
 & 料理のクオリティ & 0 & 0 & 1 & 7 \\ \cline{2-6}
 & コストパフォーマンス & 1 & 4 & 3 & 0 \\ \cline{2-6}
 & 行きたいと思うか & 1 & 2 & 4 & 1 \\ \hline
\multirow{4}{*}{うどん本陣 山田家} & 知名度 & 3 & 0 & 1 & 4 \\ \cline{2-6}
 & 料理のクオリティ & 1 & 0 & 0 & 7 \\ \cline{2-6}
 & コストパフォーマンス & 2 & 2 & 3 & 1 \\ \cline{2-6}
 & 行きたいと思うか & 0 & 3 & 3 & 2 \\ \hline
\multirow{4}{*}{ふる里うどん} & 知名度 & 5 & 1 & 1 & 1 \\ \cline{2-6}
 & 料理のクオリティ & 0 & 0 & 8 & 0 \\ \cline{2-6}
 & コストパフォーマンス & 0 & 3 & 4 & 1 \\ \cline{2-6}
 & 行きたいと思うか & 0 & 3 & 4 & 1 \\ \hline
\multirow{4}{*}{手打十段 うどんバカ一代} & 知名度 & 3 & 1 & 1 & 3 \\ \cline{2-6}
 & 料理のクオリティ & 0 & 0 & 5 & 3 \\ \cline{2-6}
 & コストパフォーマンス & 0 & 1 & 7 & 0 \\ \cline{2-6}
 & 行きたいと思うか & 0 & 1 & 5 & 2 \\ \hline
\multirow{4}{*}{さか枝うどん} & 知名度 & 2 & 0 & 2 & 4 \\ \cline{2-6}
 & 料理のクオリティ & 0 & 3 & 1 & 4 \\ \cline{2-6}
 & コストパフォーマンス & 0 & 2 & 1 & 5 \\ \cline{2-6}
 & 行きたいと思うか & 0 & 3 & 4 & 1 \\ \hline
\multirow{4}{*}{竹清 本店} & 知名度 & 4 & 3 & 0 & 1 \\ \cline{2-6}
 & 料理のクオリティ & 0 & 2 & 5 & 1 \\ \cline{2-6}
 & コストパフォーマンス & 0 & 1 & 2 & 5 \\ \cline{2-6}
 & 行きたいと思うか & 1 & 2 & 3 & 2 \\ \hline
\end{tabular}
\end{table}

	\begin{table}[H]
\centering
\caption{アンケート結果の合計}
\label{table:questionnaire:sum}
\small
\begin{tabular}{|c|c|r|r|}
\hline
項目 & 推薦方法 & 悪い\footnotemark[3] & 良い\footnotemark[4] \\ \hline
\multirow{4}{*}{知名度} & 来店者新規度 & 67 & 13 \\ \cline{2-4}
 & やみつき度 & 60 & 20 \\ \cline{2-4}
 & 万人受け度 & 35 & 45 \\ \cline{2-4}
 & 食べログ上の点数 & 49 & 31 \\ \hline
\multirow{4}{*}{料理のクオリティ} & 来店者新規度 & 11 & 69 \\ \cline{2-4}
 & やみつき度 & 11 & 69 \\ \cline{2-4}
 & 万人受け度 & 11 & 69 \\ \cline{2-4}
 & 食べログ上の点数 & 10 & 70 \\ \hline
\multirow{4}{*}{コストパフォーマンス} & 来店者新規度 & 24 & 56 \\ \cline{2-4}
 & やみつき度 & 24 & 56 \\ \cline{2-4}
 & 万人受け度 & 9 & 71 \\ \cline{2-4}
 & 食べログ上の点数 & 31 & 49 \\ \hline
\multirow{4}{*}{行きたいと思うか} & 来店者新規度 & 24 & 56 \\ \cline{2-4}
 & やみつき度 & 24 & 56 \\ \cline{2-4}
 & 万人受け度 & 20 & 60 \\ \cline{2-4}
 & 食べログ上の点数 & 28 & 52 \\ \hline
\end{tabular}
\end{table}
\footnotetext[3]{低い/絶対に行きたくない,またはやや低い/あまり行きたくないと評価された数}
\footnotetext[4]{やや高い/少し行ってみたい,または高い/かなり行ってみたいと評価された数}

	\newpage

\section{指標の評価}
アンケート結果から提案する指標について評価を行う.\par
% 食べログ上の点数の平均値の差から,提案指標による推薦では食べログの点数ランキングでは推薦されない飲食店を多く推薦できていることがわかる.
% \par
% また,
アンケートより,来店者新規度,およびやみつき度による推薦結果は知名度が低く,万人受け度,および食べログ上の点数による推薦結果は知名度が高い傾向にあるといえる.料理のクオリティに関しては推薦方法によらず,同じような結果となった.
コストパフォーマンスは万人受け度による推薦だけが突出して高い評価を得ており,その他の推薦方法では同じような結果となった.%また,もっとも行きたいと思う飲食店を推薦できたのは万人受け度であった.\par
\par
よって,提案指標は食べログでは推薦されない飲食店を推薦できているが,穴場を推薦できているといえるのは来店者新規度とやみつき度のふたつだけであり,もっとも穴場を推薦できているのは来店者新規度であり,万人受け度は穴場を推薦できていない.
%万人受け度が推薦する飲食店は,穴場ではないが,男子高専生が行きたいと思う飲食店であり,もっとも良質であると考えられる.


\chapter{考察}


\label{chap:discussion}

\section{来店者新規度についての考察}
\section{やみつき度についての考察}
\section{万人受け度についての考察}


\chapter{結言}
\label{chap:conclusion}

% 本論文では,「飲食店リピート実態&リピート要因調査」の結果から,飲食店の良質さ,および穴場飲食店とはなにかを定義し,飲食店の穴場具合を表す指標である「来店者新規度」「やみつき度」「万人受け度」を提案した.\par
 本論文では,知名度が低く,かつ良質な飲食店への推薦を実現するために,飲食店が穴場である可能性を表す指標である「来店者新規度」,「やみつき度」,および,「万人受け度」を提案した.\par
アンケートの結果により,来店者新規度がもっとも穴場を推薦できていることがわかった.また,万人受け度は穴場ではないが,良質な飲食店を推薦できていることがわかった.さらに,各指標により推薦された飲食店の業種の多様性から,提案指標は飲食店の業種の差に強いことがわかった.\par
% 今回は高松市内におけるSwarm連携ツイートから得られた来店履歴を用いて指標の評価を行ったが,他の地域や別のアプリケーションを用いて得られた来店履歴においても,提案指標が実際に有効であるのかについて検証を行う必要がある.\par
今回は高松市内におけるSwarmのチェックイン連携ツイートから得られた来店履歴を用いて指標の評価を行ったが,今後の課題として,他の地域や別のアプリケーションを用いて得られた来店履歴においても,提案指標が実際に有効であるかどうかについて検証を行うことが挙げられる.\par
% また,アンケートにおける飲食店の料理のクオリティの項目は,全ての推薦方法で同程度に高評価であった.
% やみつき度と来店者新規度の推薦結果は重複が5軒あったが,その他の推薦結果を比較すると,重複は少ないといえる.
% よって,この評価の偏りは提示した画像から判断したことに起因する可能性がある.
% したがって,評価実験のうち,料理のクオリティを調べるにあたっては,実際にその店へ行くなど,画像の提示以外の方法を取り,指標の再評価をする必要がある.


%%%%%%%%%%%%%%%%%%%%%%%%%%% 後付 %%%%%%%%%%%%%%%%%%%%%%%%%%%
\backmatter

%%%%% 謝辞 %%%%%
\chapter{謝辞}
本研究を進めるにあたって,持ち込みの研究テーマであるにも関わらず熱心に指導してくださった柿元健准教授に心より感謝いたします.また,度重なる締切の超過についてお詫び申し上げます.
\par
論文としての構成や,説明不足の箇所など,本論文の改善につながる多数の指摘をくださった副査の重田和弘教授に厚くお礼申し上げます.
\par
100項目を超えるアンケート調査に,多忙な時期にも関わらず少なくない時間を割いて協力してくださった大森君,木村君,黒川君,多田君,豊島君,溝淵君,脇坂君,渡邊君に深く感謝いたします.
\par
香川高等専門学校電気情報工学科の卒業研究の \LaTeX 版テンプレートを開発・公開してくださった北村大地助教に心より感謝いたします.


%%%%% 参考文献 %%%%%
\begin{thebibliography}{99}
  \bibitem{one}
  お
  \bibitem{two}
  こ
\end{thebibliography}


%%%%% 発表文献一覧 %%%%%

%%%%%%%%%%%%%%%%%%%%%%%%%%% 付録 %%%%%%%%%%%%%%%%%%%%%%%%%%%
\appendix

%%%%% 付録A %%%%%
% \input{appendix/appendix.tex} %TODO 付録つけることになったら書く

\end{document}
