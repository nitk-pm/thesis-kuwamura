%#!pdfpLaTeX
%
% 北村研究室用卒業論文・特別論文のTeXテンプレートファイル
% 本ファイルは非公式であり,表紙とアブストに関しては下記で公開されているワードの
% テンプレートを利用して作成したものが公式であるので,表紙とアブストはPDFにして
% 差し替えること.
% https://www.kagawa-nct.ac.jp/EE/local/index.html (学内限定アクセス)
%
% 2020年1月17日 北村大地作成
%

%%%%%%%%%%%%%%%%%%%%%%%%%%% 論文情報 %%%%%%%%%%%%%%%%%%%%%%%%%%%
%%%%% テンプレート選択 %%%%%
\documentclass[honka]{nitkcthesis}%卒論(本科5年)日本語用

%%%%% タイトル %%%%%
\title{\underline{全ユーザの来店履歴を用いた}\\\underline{飲食店推薦指標の提案}}
%\titlewidth{}% タイトル幅 (指定するときは単位つきで)

%%%%% 著者 %%%%%
\author{name}
\eauthor{name}% Copyright表示で使われる

%%%%% 指導教員名 %%%%%
\supervisor{name}% 1つ引数をとる (役職まで含めて書く)


%%%%% 提出年月 %%%%%
\date{令和3年 3月 31日}

%%%%% \usepackage等のプリアンブル宣言(macros.texに記載) %%%%%
\usepackage{bm}
\usepackage{amsmath, amssymb}
\usepackage[dvipdfmx]{color}
\usepackage[dvipdfmx]{graphicx}
\usepackage{tabularx}
\usepackage{booktabs}
\usepackage{multirow}
\usepackage{setspace}
\usepackage{amsthm}
\usepackage[caption=false]{subfig}
\usepackage[numbers,sort]{natbib}

\theoremstyle{definition}
\newtheorem{theo}{定理}[chapter]
\newtheorem{defi}{定義}[chapter]
\newtheorem{lemm}{補題}[chapter]
\renewcommand{\proofname}{\textbf{証明}}

%% definition
\newcommand{\J}{\mathrm{j}}
\newcommand{\diag}{\mathop{\mathrm{diag}}}

\newcommand{\mtr}[1]{#1^{\mathsf{T}}}
\newcommand{\ctr}[1]{#1^{\mathsf{H}}}
\newcommand{\inv}[1]{#1^{-1}}
\newcommand{\cinv}[1]{#1^{-\mathsf{H}}}
\newcommand{\tinv}[1]{#1^{-\mathsf{T}}}
\newcommand{\conj}[1]{#1^*}

\newcommand{\tbm}[1]{\tilde{\bm{#1}}}
\newcommand{\tsf}[1]{\tilde{\mathsf{#1}}}

\newcommand{\vw}{\bm{w}}
\newcommand{\mW}{\bm{W}}
\newcommand{\vwhat}{\widehat{\bm{w}}}
\newcommand{\mWhat}{\widehat{\bm{W}}}

\newcommand{\hhat}{\widehat{h}}
\newcommand{\rhat}{\widehat{r}}

\newcommand{\argmax}{\mathop{\mathrm{arg~max}}\limits}
\newcommand{\argmin}{\mathop{\mathrm{arg~min}}\limits}

\renewcommand{\Re}{\mathop{\mathrm{Re}}}
\renewcommand{\Im}{\mathop{\mathrm{Im}}}

\newcommand{\unit}[1]{~\mathrm{#1}}
\newcommand{\Unit}[1]{~\mathrm{\left[#1\right]}}

\renewcommand{\qedsymbol}{$\blacksquare$}

\bibliographystyle{IEEEtran}

\makeatletter
\def\bstctlcite{\@ifnextchar[{\@bstctlcite}{\@bstctlcite[@auxout]}}
\def\@bstctlcite[#1]#2{\@bsphack
\@for\@citeb:=#2\do{%
\edef\@citeb{\expandafter\@firstofone\@citeb}%
\if@filesw\immediate\write\csname #1\endcsname{\string\citation{\@citeb}}\fi}%
\@esphack}
\makeatother


\begin{document}
\bstctlcite{IEEEexample:BSTcontrol} % BibTeXのIEEEtranで同一著者の横線表示を防止

\maketitle% タイトル生成

%%%%%%%%%%%%%%%%%%%%%%%%%%% 前文 %%%%%%%%%%%%%%%%%%%%%%%%%%%
\frontmatter

%%%%% English title %%%%%
\etitle{proposal of metrics for restaurant recommendation using all users' visitation history}

%%%%% Abstract %%%%%
\eabstract{
English abstract goes here.
}

%%%%% 概要 %%%%%
\abstract{
日本語の概要をここに記述.
}

\keywords{A, B, C}

\makeseparatedabstract
%\makeabstract
%%%%% 目次 %%%%%
%\tableofcontents % ページ番号を削除しない目次
%----- ページ番号を削除した目次 -----%
{\makeatletter
\let\ps@jpl@in\ps@empty
\makeatother
\pagestyle{empty}
\tableofcontents
\clearpage}
%---------------------------------%

%%%%%%%%%%%%%%%%%%%%%%%%%%% 本文 %%%%%%%%%%%%%%%%%%%%%%%%%%%
\mainmatter

\chapter{緒言}
\label{chap:intro}
身近な飲食店推薦サービスである食べログ[N]やRetty[N]では,有志のユーザによるレビューをもとに飲食店の良し悪しを評価する,集合知による推薦が行われている.
しかし,この方式では,飲食店の質に関わらず,知名度が低い場合は高い点数を得られない.
特に食べログでは,評価の数が少ない場合,点数が低くなることが明言されている[N].
新規オープンした飲食店や熱心な集客活動を行っていない飲食店は,その質によらず,知名度や評価数の点で不利である.
したがって,知名度が高い店は推薦されやすく,多数の評価を得られるのに対し,知名度が低い店は推薦されにくく,なかなか評価を得ることができないという問題が生じる
.この問題を解消するため,良質だが知名度が低い穴場飲食店への推薦が求められている.

飲食店が穴場であるという根拠になるのが,少数のリピーターによる頻繁な来店である.
本研究では,穴場飲食店とは「少数のユーザが頻繁に通い,その他のユーザはその存在に気づいていない飲食店」であるという仮定のもと,来店履歴データを利用した指標を考案する.
また,本研究では,リクルートライフスタイルによる「飲食店リピート実態&リピート要因調査」[N]をもとに,本研究における「飲食店の良質さ」を定め,指標の評価に用いる.

さらに,近年では,スマートフォンの普及に伴い,多くの人々がSNS上で手軽に様々な情報を共有するようになった.
その中でも,位置情報を共有するSNSであるSwarm[N]は,スマートフォンのGPS機能を使って,自分が現在どこにいるかを「チェックイン」という形で記録,共有できる.
また,Twitter[N]と連携し,チェックイン情報をツイートすることで,ライフログを公開データとして記録することができる.
本研究では,Twitter上のSwarmのチェックインに連携したツイートをもとに来店履歴データを作成し,高松市内の飲食店に対して指標値を実際に計算する.
得られた指標値をもとに推薦された飲食店と既存サービスにより推薦された飲食店に対して,良質さと知名度をアンケートで比較し,指標が穴場を推薦できているかの評価を行う.


\chapter{結言}
\label{chap:conclusion}

% 本論文では,「飲食店リピート実態&リピート要因調査」の結果から,飲食店の良質さ,および穴場飲食店とはなにかを定義し,飲食店の穴場具合を表す指標である「来店者新規度」「やみつき度」「万人受け度」を提案した.\par
 本論文では,知名度が低く,かつ良質な飲食店への推薦を実現するために,飲食店が穴場である可能性を表す指標である「来店者新規度」,「やみつき度」,および,「万人受け度」を提案した.\par
アンケートの結果により,来店者新規度がもっとも穴場を推薦できていることがわかった.また,万人受け度は穴場ではないが,良質な飲食店を推薦できていることがわかった.さらに,各指標により推薦された飲食店の業種の多様性から,提案指標は飲食店の業種の差に強いことがわかった.\par
% 今回は高松市内におけるSwarm連携ツイートから得られた来店履歴を用いて指標の評価を行ったが,他の地域や別のアプリケーションを用いて得られた来店履歴においても,提案指標が実際に有効であるのかについて検証を行う必要がある.\par
今回は高松市内におけるSwarmのチェックイン連携ツイートから得られた来店履歴を用いて指標の評価を行ったが,今後の課題として,他の地域や別のアプリケーションを用いて得られた来店履歴においても,提案指標が実際に有効であるかどうかについて検証を行うことが挙げられる.\par
% また,アンケートにおける飲食店の料理のクオリティの項目は,全ての推薦方法で同程度に高評価であった.
% やみつき度と来店者新規度の推薦結果は重複が5軒あったが,その他の推薦結果を比較すると,重複は少ないといえる.
% よって,この評価の偏りは提示した画像から判断したことに起因する可能性がある.
% したがって,評価実験のうち,料理のクオリティを調べるにあたっては,実際にその店へ行くなど,画像の提示以外の方法を取り,指標の再評価をする必要がある.


%%%%%%%%%%%%%%%%%%%%%%%%%%% 後付 %%%%%%%%%%%%%%%%%%%%%%%%%%%
\backmatter

%%%%% 謝辞 %%%%%
\chapter{謝辞}
本研究を進めるにあたって,持ち込みの研究テーマであるにも関わらず熱心に指導してくださった柿元健准教授に心より感謝いたします.また,度重なる締切の超過についてお詫び申し上げます.
\par
論文としての構成や,説明不足の箇所など,本論文の改善につながる多数の指摘をくださった副査の重田和弘教授に厚くお礼申し上げます.
\par
100項目を超えるアンケート調査に,多忙な時期にも関わらず少なくない時間を割いて協力してくださった大森君,木村君,黒川君,多田君,豊島君,溝淵君,脇坂君,渡邊君に深く感謝いたします.
\par
香川高等専門学校電気情報工学科の卒業研究の \LaTeX 版テンプレートを開発・公開してくださった北村大地助教に心より感謝いたします.


%%%%% 参考文献 %%%%%
\begin{thebibliography}{99}
  \bibitem{one}
  お
  \bibitem{two}
  こ
\end{thebibliography}


%%%%% 発表文献一覧 %%%%%

%%%%%%%%%%%%%%%%%%%%%%%%%%% 付録 %%%%%%%%%%%%%%%%%%%%%%%%%%%
\appendix

%%%%% 付録A %%%%%
% \input{appendix/appendix.tex} %TODO 付録つけることになったら書く

\end{document}
